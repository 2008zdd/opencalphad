\documentclass[12pt]{article}
\usepackage[latin1]{inputenc}
\usepackage{graphicx,subfigure}
\topmargin -1mm
\oddsidemargin -1mm
\evensidemargin -1mm
\textwidth 165mm
\textheight 220mm
\parskip 2mm
\parindent 3mm
%\pagestyle{empty}

\begin{document}
\begin{center}
{\Large \bf Some important facts about Open Calphad in general and
  this test release in particular.

}

Bo Sundman, \today

http://www.opencalphad.org or

opncalpad repoitory at http://www.github.com

bo.sundman@gmail.com
\end{center}

{\large \bf Please be aware that this software is a test version.
  Your feedback about problems and errors is important to make it
  better.

}

OC cannot replace your favourite thermodynamic software today or
tomorrow but the main advantage is that you have access to the source
code and can (with some efforts) add or fix things yourself that you
are missing in your favourite software.

There is a review paper describing OC:

{\bf OpenCalphad - a free thermodynamic software}, Bo Sundman, Ursula
R Kattner, Mauro Palumbo and Suzana G Fries (Open Access) Integrating
Materials and Manufacturing Innovation (2015) {\bf 4:}1 DOI
10.1186/s40192-014-0029-1

and a special paper about the minimization algorithm:

{\bf The implementation of an algorithm to calculate thermodynamic
  equilibria for multi-component systems with non-ideal phases in a
  free software}, Bo Sundman, Xiao-Gang Lu and Hiroshi Ohtani,
Computer Materials Science, {\bf 101} (2015) 127-137.

\section{General information}

\begin{itemize}
\item The code is written in the new Fortran standard and requires a
  compiler like GNU Fortran 4.8 or similar.

\item The last release of the source code can be found as a zip file
  on http:://www.opencalphad.org.

  The last development version can be obtained as a zip file at the
  opencalphad repository on the http://www.github.com.  On that site
  there is also a wiki for information, discussion and questions.

\item Users with Windows can compile and link the program using the
  file ``linkmake.txt'' which you should rename to ``linkmake.cmd'' in
  order to compile and link an executable program from the source
  code.

  If you are using a UNIX system there is a Makefile to compile and link
  the program.

  In both cases the program will be called ``ocj'' where j is the
  version number.

\item For the graphics you must download and install the GNUPLOT
  software.  That is freely available from the web.

  Make sure your PATH includes the directory with the GNUPLOT program.
  If you do not know how to set your PATH ask an expert.

\item There are several directories provided in the zip file, the
  source code is in the directories ``minimizer, models, numlib,
  stepmapplot, userif'' and ``utilities''.

\item The documentation of the source code is in the directory
  ``documentation'' in two PDF files ``gtp3'' and ``minpack5'' for the
  model and minimizer package.  There is also a very rudimentary user
  guide ``ochelp''.  More informal guides are ``news-OC2'' and
  ``OC3-commands''.

\item The user guide is also available in the file ``ochelp.hlp'' in
  LaTeX format used for the on line help.  The user guide has not been
  updated since version 1.

\item The ``macro'' directory has examples for a variety of
  calculations.  There are subdirectories created for testing
  different versions but as the command interface changes those
  provided with this version will normally not work with previous
  versions.

\item The TQ3lib directory has examples for using the software
  interface for Fortran and C/C++.

\item Contributions of new and improved source code are welcome.  You
  can do this using the github repository.  Contact Bo Sundman if you
  want to know more.

\item The command line interface has a ``VAX/VMS'' flavour which
  reflects the age of the developer.  It means the commands are
  ``verbs'' like {\em set, list, calculate, enter} etc.  After the
  verb several objects are usually possible like {\em
    set~condition/status} etc.  There is some redundancy so the same
  effect can sometimes be achieved by different combinations of verbs
  and objects.

  Each command and subcommand can be abbreviated.

\end{itemize}

\section{Some short hints to the user interface}

See also news-OC2 and OC3-commands.  The information in these (as well
as this) file is not always up-to-date.  You are welcome to inform me
about errors an inconsitencies.  If you do not like the current user
interface you are welcome to implement your own.

\subsection{Entering and manipulating thermodynamic data}

In this and the following sections you can find a brief and
unstructured summary of the commands that may be useful when trying to
get aquantied to OC.

\begin{itemize}
\item Thermodynamic data can be read from an unencrypted TDB file {\em
  read~tdb~``filename''} or entered interactively, see the {\em enter}
  commands.  You can specify which elements you want from the TDB file
  but not select the phases.  Phases you do not want you can suspend
  from the calculations with the command {\em
    set~status~phase~...~=~suspend}.

  If you read thermodynamic from a file and already have data in the
  program you have to confirm that you want to overwrite them.  You can
  remove all the thermodynamic data in the program by the command {\em
    new~YES}.

\item There are some exception from how data are entered into OC
  comparared to Thermo-Calc, most important perhaps the
  ``partitioning'' of a phase into an ordered and disordered part.  OC
  has implemeneted this in a different way and at present it is not
  always possible for OC to read such partitioned data from a TDB file
  without editing.

\item You can save the thermodynamic data and results from a
  calculation on an unformatted file {\em save~UNFORM~``filename''}
  and this can be read back into the program {\em
    read~UNFORM~``filename''}.  This is fragile.  The DIRECT (random
  access) format is not yet implemented.

\item You can save the thermodynamic data using the TDB format by a
  command {\em list~data~TDB~filename}.  The TDB format used by OC can
  with some exceptions be read also by other thermodynamic software.

\item You can list the thermodynamic data on the screen with the {\em
  list~data} command.  You can list on a file by using the switch {\em
  list~/output=filename ...}.  If you want to write a TDB file use the
  command {\em list~data~TDB}.
  
\item You can enter thermodynamic data from a macro file as well as
  commands for calculations.  In order to document errors or problems
  please send a complete macro file with data and commands reproducing
  the error.  

  You can generate a macro file interactivly by setting a log file
  {\em set~log~``filename''} and then (after some editing) use the log
  file as a macro with the extention OCM.  The log file has the
  extention LOG.

\end{itemize}

\subsection{Setting conditions}

\begin{itemize}
\item To calculate you must set conditions on the external variables
  like temperature, T, pressure, P etc.  The number of conditions must
  be equal to the number of components plus 2 (Gibbs phase rule).

\item Setting conditions is very similar to the Thermo-Calc software.
  Each condition is set separately by the command is {\em
    set~cond~``state variable''~=~``value''}.  The safest set of
  conditions for calculating an equilibrium, i.e. which has most
  chance to converge, is to set values on T, P and N(element),
  i.e. the total amount of each element.  The table at the end gives a
  list of available state variable symbols and their meaning.

\item It is also possible to set conditions on chemical potentials,
  activities and that a phase is stable (fix).  See the macro file
  examples how this is done.  Note that some commands are fragile and
  they may also change between releases of the test versions of the OC
  software, depending on new ideas and suggestions by users.

\item The intention is that you should be able to combine any set of
  conditions to calculate the equilibrium, i.e. you should be able to
  combine conditions on mole fractions, mass fractions, fix phases,
  chemical potentials, enthalpy, volume etc.  In a future release the
  plan is to allow expressions of conditions, not just a single state
  variable.

\item For FactSage users there is a command {\em set~input\_amount}
  which allows the user to set the overall composition by specifying
  the amount of different species in the system.

\end{itemize}

\subsection{Calculation}

\begin{itemize}
\item The command {\em calculate~equilibrium}, which can be
  abbreviated {\em c~e}, tries to calculate the equilibrium.  As the
  minimizer needs a guess of stable phases and their start
  constitution, the OC software tries first to invoke a global grid
  minimizer to find such a guess.  The grid minimizer requires that
  the conditions are $T, P$ and overall composition (N, B, x or w).

\item If you want to provide yourself a guess of the set of stable
  phases and consitutions yourself you can use the {\em
    amend~const~phase~``name''~``amount''~``constitution'' $\cdots$}
  followed by the command {\em calculate~no\_global}, abbreviated {\rm
    c~n}.

\item The grid minimizer that calculates start points for the general
  minimizer is powerful but a bit primitive.  If you have ideas how to
  improve it you are welcome to provide advice or code.

\item For this test version you should always check the results with
  you favourite thermodynamic software.  Please give documented
  feedback on any differences.

\item If the calculation does not converge directly try to use the
  command {\em ``c~n''} two or more times to continue to iterate from
  the set of phases you have.  The {\em ``c~e''} command runs the grid
  minimiser and, therefore, will normally give the same result each
  time.

\item You can also try to change the number of iterations and
  convergece criteria with the command

  {\em set~num~``max~iterations''~``convergce~criteria''}.  Default
  values of these are 500 and 10$^{-6}$.

\item If everyting fails try to simplify the conditions for a first
  calculation.  The simplest are values of $T, P$ and $N(\rm
  component)$. When that has converged change the conditions one by
  one to those you are interested in and for each change calculate
  without the grid minimizer, i.e.  using {\em ``c~n''}.  Calculations
  at temperatures and compositions where the system is single phase
  have a higher chance of success.  The algorithm to change the set of
  stable phases is fragile and is still being fine tuned.

\item Phases with miscibility gaps can be a problem.  The grid
  minimizer should detect such and automatically add new composition
  sets for phases that can be stable with two different compositions.
  Composition sets created automatically have the suffix AUTO.

\item The user can also add composition sets manually with the command
  {\em amend~phase~``name''~comp\_set }.  You can the add a prefix
  and suffix (max 4 letters) to the phase name (for example the cubic
  carbide treated as a composition set to the FCC phase can have the
  prefix MC so the full name is MC\_FCC.

  Composition sets have a number after the phase name and the letter
  \#, like LIQUID\#2.  The user can use both pre- and suffixes or set
  number to specify the composition set.  A phase can have up to 9
  composition sets.

\item When entering a composition set manually you are asked for a
  default constitution.  After a calculation the software will try to
  match a calculated constitution of a phase with several composition
  sets to the set with the closest default constitution.

\item To set the default constitution of the first compostion set use
  {\em set~phase~``name''~default\_constitution~$\cdots$ }.

\item There are some more option to calculate things:
  \begin{itemize} 
  \item {\em calculate~all} will be used when OC can do assessments,
  \item {\em calculate global\_gridmin} will only use the gridminimizer
    and not the final iterative routine,
  \item {\em calculate~phase} calculates the Gibbs energy and possibly
    derivatives for a specified phase and composition at the current $T$
    and $P$.
  \item {\em calculate~symbol} will calculate the value of one or more
    symbols.
  \item {\em calculate~tp\_funs} will calculate the value and the
    first and second derivatives with respect to $T$ and $P$ for all
    tp functions (used for the model parameters).
  \item {\em calculate~transition~``phase''~``condition''} calculates
    the value of the ``condition'' when the phase is just stable, for
    example the melting $T$ if the ``phase'' is LIQUID.
  \end{itemize}
\end{itemize}

\subsection{Setting the status of phases: entered, fixed, dormant, suspend}

To set the status of a phase to be fixed is treated as a condition.

\begin{itemize}
\item By default all phases have the status ENTERED.  You can suspend
  phases you do not want by {\em
    set~status~phase~``name''~$\cdots$~=~``suspend''}.

\item To specify that a phase must be stable you can set its status to
  fix, the command is

  {\em set~status~phase~``name''~=~``fix''~''amount''}.

\item The dormant status means the phase cannot be stable but its
  driving force is calculated.  If this is positive the phase wants to
  be stable and the calculated equilibrium is metastable.
\end{itemize}

\subsection{Listing many things}

Most listing are written on the screen but when output obviously
should be written on a file the program will ask for a file name.  If
you want the output from other listings to be written on a file you
can use the switch {\em /output=filename} or {\em /append=filename}
directly after the command.  For example {\em list~/out=result~r,,,}.
The extension of the output file will be DAT.  The output file will be
reset to the screen after each command.

There is also a SAVE command mainly intended for saving on a file.
For example {\em save tdb filename} will write all the thermodynamic
data in the OC TDB format on the file with the extention TDB.  This
can be read back later as a TDB file.

\begin{itemize}
\item Listing results after a calculation is done by {\rm list result
  ``option''}.  This is the default list and can be abbreviated {\rm
  l,,}.  This list will always include the current conditions and some
  global results and one line for each component.  The options are at
  present:
  \begin{itemize}
    \item 1 for list of stable phases with amount and composition in
      mole fractions.
    \item 2 for list of stable phases with amount, composition in
      mole fractions and constitution.
    \item 3 same as 1?
    \item 4 for list of stable phases with amount and composition in
      mass fractions.
    \item 5 same as 4?
    \item 6 for list of stable phases with amount and composition in
      mass fractions and the constitution
    \item 7 for list of all phases with amount and composition in
      mass fractions.
    \item 8 for list of all phases with amount and composition in
      mass fractions and constitution.
    \item 9 for list of all phases with amount and composition in
      mole fractions and constitution.
  \end{itemize}
\item The {\em list~short~A} gives one line for each component,
  species and phase.  In this list you may note the use of the ``phase
  tuple'' to indicate composition sets.  A phase tuple has two values,
  one is the phase number, the other the composition set number.  The
  first composition set of a phase has the same tuple number as the
  phase number.  As composition sets can be added and deleted the
  tuple number of a higher composition set may change.
\item The {\em list~short~P} gives one line for each phase, separated
  in stable, entered and dormant, listed in order of increasing
  instability (driving force).
\item The {\em list~bibliography} list all or a single bibliographic
  reference.  Bibliograpic references are entered when entering
  parameters.  The {\em amend~biblio} allows interactiving amending of
  the bibliographic item.
\item The {\em list~data} lists all thermodynamic data with
  bibliographic references.  Note this output is not the TDB format.
\item The {\em list~equilibria} list all entered equilibria (not
  any results).  By default there is only one equilibrium but during step
  and map each node point is entered as an equilibria.  The user can
  also enter equilibria manually.
\item The {\em list~model\_param\_id} list all symbols for which one
  can enter parameters.  By default the G is Gibbs enrgy, TC the
  combined Curie and Neel temperature etc.
\item The {\em list~phase} has several options to list various data
  for a phase.
\item The {\em list~state\_variables} lists individual values of state
  variables.  Terminate with an empty line.
\item The {\em list~symbols} list the symbols and their expressions.
  There two predefined symbols, R and RT.  The user can enter any
  number of additional symbols as functions of state variables or
  other symbols.  To calculate (and list) the value of a symbol use
  {\em c~sym}.
\item The {\em list~tp\_symb} lists the functions entered for the
  model parameters.  To calculate their current values use {\em c~tp}.
\end{itemize}

\subsection{Calculating and plotting diagrams}

This part is unfinished and very rudimentary.

\begin{itemize}
\item You can calulate property diagram (one independent axis
  variable) or phase diagrams (two or more idependent axis variables).
  The command to set an independent axis is 

  {\em set~axis~1~``state~variable''~``min value''~``max
    value''~``increment''}.  The state variable used must be a
  condition.  Axis 1 and 2 are implemented.

\item With one axis the {\em step} command will step from the current
  equilibrium to the max and min value.

\item With two axis the command {\em map} will follow ``zero phase
  fraction'' (ZPF) lines within the limits of the axis by replacing
  one of the axis consition with a phase fix with zero amount.  In
  this way phase diagrams for any number of components can be
  calculated.  While following such a line other phases may become
  stable or disappear and this generates a node point where several
  lines meet.

\item Plotting of diagrams relies on GNUPLOT, a free software you can
  install from the web.  The command {\em plot} asks for two axis
  variables and generates pairs of values from the resluts calculated
  by step or map commands.

  There are several options to the plot command that are demonstrated by
  the macro files.

\item Macro files are easily created by using a log file while running
  OC interactivly.  On the log file all input (as well as default
  values) are written.  Then you can edit this and finish it with the
  {\em set interactive} command to give command back to the keyboard.

  When you report problems or errors always sent a macro file that
  reporoduces the error.  Please try to find the simplest case that
  contains the error.
\end{itemize}

\section{A summary of state variables.}

The state variables are very important for setting conditions and
listings.  Those recognized by OC are given in
table~\ref{tab:statevar}.

\begin{table}[!h]
\caption{A very preliminary table with the state variables and their
internal representation.  Some model parameter properties are also
included.}\label{tab:statevar}
\begin{tabular}{|llccll|}\hline
Symbol & Id & \multicolumn{2}{c}{Index} & Normalizing & Meaning\\
       &    & 1 & 2                     &  suffix     & \\\hline
\multicolumn{6}{|c|}{Intensive properties}\\\hline
T      & 1  & -         & -    & - & Temperature\\
P      & 2  & -         & -    & - & Pressure\\
MU     & 3  & component & -/phase  & - & Chemical potential\\
AC     & 4  & component & -/phase  & - & Activity\\
LNAC   & 5  & component & -/phase  & - & LN(activity)\\\hline
\multicolumn{6}{|c|}{Extensive and normallized properties}\\\hline
U      & 10 & -/phase\#set & - & - & Internal energy for system\\
UM     & 11 & -/phase\#set & - & M & Internal energy per mole\\
UW     & 12 & -/phase\#set & - & W & Internal energy per mass\\
UV     & 13 & -/phase\#set & - & V & Internal energy per m$^3$\\
UF     & 14 & phase\#set   & - & F & Internal energy per formula unit\\
Sz     & 2z & -/phase\#set & - & - & entropy\\
Vz     & 3z & -/phase\#set & - & - & volume\\
Hz     & 4z & -/phase\#set & - & - & enthalpy\\
Az     & 5z & -/phase\#set & - & - & Helmholtz energy\\
Gz     & 6z & -/phase\#set & - & - & Gibbs energy\\
NPz    & 7z &  phase\#set & - & - & Moles of phase\\
BPz    & 8z & phase\#set & - & - & Mass of phase\\
Qz     & 9z & phase\#set & - & -  & Stability of phase\\
DGz    & 10z & phase\#set & - & -  & Driving force of phase\\
Nz     & 11z & -/phase\#set/comp & -/comp & -  & Moles of component\\
X      & 111 & phase\#set/comp & -/comp & 0  & Mole fraction\\
X\%    & 111 & phase\#set/comp & -/comp & 100 & Mole per cent\\
Bz     & 12z & -/phase\#set/comp & -/comp & -  & Mass of component\\
W      & 122 & phase\#set/comp & -/comp & 0 & Mass per cent\\
W\%    & 122 & phase\#set/comp & -/comp & 100 & Mass per cent\\
Y      & 130 & phase\#set & const\#subl & -& Constituent fraction\\\hline
\multicolumn{6}{|c|}{Some model parameter identifiers}\\\hline
TC     & - & phase\#set & - & - & Curie temperature\\
BMAG   & - & phase\#set & - & - & Aver. Bohr magneton number\\
MQ\&X  & - & phase\#set & constituent X & - & Mobility\\
THET   & - & phase\#set & - & - & Debye temperature\\\hline
\end{tabular}
\end{table}

The state variables in the user interface have their common symbols,
$T$ for temperature, $P$ for pressure, $N$ for the total amount of
moles, ``$N$(element)'' for the amount of moles of a component,
``$X$(element)'' for the mole fraction ``$MU$(element)'' for the
chemical potential, ``$AC$(element)'' for the activity.  The symbol
$B$ is used for the total mass (copied from the Thermo-Calc software),
``$B$(element)'' for the mass of an element and ``$W$(element)'' for
the mass fraction.  There are many more state variables like $H$, $G$
etc, see the table, but not all of them can be used as conditions.

\section{Phase status}

You can specify that a phases should be stable by the command {\em set
  status phase $\cdots$}.  For example to calculate the melting point
of an alloy after specifying the composition and making a calculation
at fixed T and P, you can give the commands {\em set cond T=none; set
  status liquid=fix 0; c n}.  (The commands must be given on separate
lines).  You can also use the command {\em calculate transition
  $\cdots$}.

\section{Manipulating the source code}

The OC software is provided with a GNU license which means that you
have the source code and can use it and modify it as you wish as long
as you do not try to make money of it.  If you want to include the OC
software in a commercial program you must contact the copyright
owners.

There is a fairly extensive documentation of the source code in the
directory ``documentationupdate'' and if you look at the code itself
there are some comments there too.  I have spent a lot of effort to
make the datastructures general and flexible to handle multicomponent
and multiphase systems.  But there was quite a lot of redundancy
introduced during the development that eventually will be removed.
The set of subroutines is less structured and one problem has been
that this code was my first attempt to use the new Fortran standard so
there are probably many things that can be made simpler.  You are
welcome to point out where this can be done.  The hope is to have a
new release before the end of 2015 with an improved version of this
part of the code and some more thermodynamic models and facilities for
conditions and generating diagrams will be available.

As I have understood the data structures (TYPE) in the new Fortran
standard is more or less identical to those used in C++ so it should
not be too difficult to combine code written in these languages.

\section{Application software library, TQ}

There is a software intrface following the TQ standard to make it
possible to use OC from application programs.  The subroutines in this
interface can be found in the directory ``TQ3lib'' together with a few
test program.  To compile and run these the OClib.a file and some of
the ``mod'' files must be copied to this directory, there is a readme
file that should explain how to use these.

There is a Fortran version of the TQ library and also an isoC version
that can be called from C++/C routines.  The isoC binding makes it
possible to access data directly from the datastructures defined
inside the OC software.  There are some test programs for this also.

Use only subroutines in the TQ library to access the OC software, do
not call directly subroutines inside the OC code as they may not be
available or have different functionallity in a future release.  If
you miss some routines please sent a message or discuss this on the
github wiki.

\bigskip

{\large \bf Have fun and help make OC useful!}

\end{document}

