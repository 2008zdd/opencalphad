\documentclass[12pt]{article}
\usepackage[latin1]{inputenc}
\usepackage{graphicx,subfigure}
\topmargin -1mm
\oddsidemargin -1mm
\evensidemargin -1mm
\textwidth 165mm
\textheight 220mm
\parskip 2mm
\parindent 3mm
%\pagestyle{empty}

\begin{document}

\begin{center}
{\Large \bf An overview of the Open Calphad software}

\bigskip

Bo Sundman \today

\end{center}

\section{Introdution}

The aim of the Open Calphad (OC) initiative\cite{14Sun} is to provide
a high quality multicomponent thermodynamic software for scientific
applications as well as for research and developments.  It has a GNU
license and the hope is this will make it interesting for the
development of new thermodynamic models, new algorithms for
minimization as well as many new applications of thermodynamic
calculations in materials science and process developments.

The two basic parts are:

\begin{itemize}
\item the General Thermodynamic Package (GTP) which include the
  Compound Energy Formalism (CEF)\cite{01Hil,07Luk} and the partially
  ionic two-sublattice liquid model (I2SL)\cite{85Hil,07Luk}.  The CEF
  includes a large range of models from the gas with molcules, liquid
  with associates, crystalline phases with sublattices, ions etc.  The
  magnetic contribution to the Gibbs energy is described withe the
  empirical model proposed by Inden\cite{81Ind}.  In CEF the
  configurational entropy is bases on the assumpton that the
  constituents on each set of sites are rsndomly distributed.
\item the Hillert Mimimizer by Sundman (HMS) package with an
  implementation of the algorithm proposed by
  Hillert~\cite{81Hil,15Sun} to find the equilibrium for kinds of
  external conditions by using Lagrangian multipliers.  The user can
  also prescribed some or all stable phases and the software can
  handle changes of the set of stable phases during iterations.  It
  requires that the model package provides the Gibbs energy and first
  and second derivatives of the Gibbs energy with respect to $T, P$
  and all constituents for each phase.
\end{itemize}

These two packages have separate documention of their data structure
and subroutines.

OC has a simple command user interface and can read thermodynamic data
from files with the TDB format\cite{TDB}.  There is a rudimentary user
guide also available online.  Data can be entered and amended
interactivly and there is a log and macro facility.  There are also
three untility packages, the METLIB package for general utilities, the
TPFUN package for handling functions of temperature and pressure and
NUMLIB, a numerics package for inverting matricies and solving systems
of linear equations.  It is all written in or converted to the new
Fortran standard (1995/2008).

The first application package using the GTP and HMS is the
step/map/plot (SMP) package for calculation property and phase
diagrams and the OC-TQ software interface for general applications,
for example simulations of phase transformations also requiring
kinetic data.  The OC initiative will continue at least until there is
an assessment module for evaluating thermodynamic model
parameters from experimental data.

The development of a Graphical User Interface (GUI) is invited but not
part of the OC initiatve.

\section{Implementation of new models and minimization algorithms}

Models based on CEF and the I2SL model are described in detail in the
book by Lukas\cite{07Luk}.  There is a very general data structure
defined in the GTP package for handling model parameters including
sublattices and various aditional contributions like magnetic
transitions.  The data structure can also handle other types of data
that may depend on the phase and $T, P$ and the phase constitition,
for example mobilities, elastic constants, Curie temperature etc.  It
may be useful also for the development of new models.  The GTP
contains a routine to calculate the Gibbs energy and the first and
second derivatives with respect to $R, P$ and all constituents for the
implemented models.  State variables and conditions on state variables
are also part of the model package and are used by the HMS package
when calculating the equilibrium.  On the list of models to be added
there are:

\begin{itemize}
\item the corrected quaichemical model\cite{08Hil}
\item the tetrahedron model of the Cluster Variation Method\cite{51Kik}
\end{itemize}

The hope is that the availability of OC will simplify and encourage
the development of new models based of better physics including strain
and stress and also for high pressure applications.

It would also be interesing to implement other minimization procedures
that may be adopted for special models.

\section{The step/map/plot package}

There is no separate documentation for this package at present so the
priciples are briefly explained here.

\subsection{The step procedure}

For a property diagram the user specifies conditions for a single
equilibrium calculation and then selects one of the conditions as axis
variable.  With the STEP command the software will calculate
equilibria between a maximum and minimum value at fixed intervals.
Whenever there is a change of the set of stable phases the axis value
for this change is calculated exactly.  The complete results are saved
for each calculated equilibrium and the user can later plot any
property or derived function from the stored equilibria.

There is a special version of the STEP command allowing each phase to
be calculated separately.  This is useful for plotting the Gibbs
energy or enthalpy curves, along the axis.

Other special STEP commands may be implemented in the future, for
example to simulate a Scheil-Gulliver solidification process.

\subsection{The map procedure}

In the map procedure the user first specifies all conditions needed
for calculating a single equilibrium.  Then two of these conditions
are selected as axis variables and with the MAP commad the software
calculates two-dimensional sections or projections with lines with
different sets of stable phases.  At present only two axis are
allowed.  The map software replaces all axis conditions except one
with a phase prescribed to be stable with zero amount and then follows
this line by incrementing the remaining axis with an active condition.
If the curvature is line requires it may change the axis with active
condition.  If there is a phase change the software generates a node
point and three or more lines with different set of prescribed stable
phases will exit from each mode point.  If a node point is reached
that has already been calculated an exit from that node point is
eliminated.

During the step and map calculation all equilibria are saved and any
state variable can be used to plot the stored result.  It is possible
to start mapping from different equilibra and overlay the results.
But it is not yet possible to save the results on a file.

\subsection{The plot procedure}

The plot procedure at present can only plot diagram with a single
variable on one axis, for example a potential like $T$ or an overall
composition.  This means isothernal sections cannot be plotted but
that will be implemented in the next release.  The plot routine
generates a table that is plotted by calling GNUPLOT.

\section{Application software interface, OC-TQ}

Application programs using OC should not use directly the
datastructures and subroutines inside GTP or HMS to avoid that they
need to be modified whenever there is a change.  Instead there is an
application interface called OC-TQ (Thermodynamic Query system) which
will (eventually) give access to all necessary facilities, also
callable from other languages like C or Phyton.

As the OC-TQ interface is in the development stage the use of the
current OC-TQ interface is not stable and may need modifications in
the future.  And many facilities are still not implemented.  There
will be a separate documentation of OC-TQ.

\section{Summary}

A general multicomponent software has been developed that can
calculate single equilibria and property diagrams.  The phase diagram
module can calculate most binary phase diagram but fails in most cases
when dealing with ternary or higher systems.  A revision of the data
structure will be made in the next relase to improve the phase diagram
calculation.

The first version of the software was released in March 2013 and the
most resecent version is available on the wed\cite{ocweb}.  A
development version of the furure relase is available from a
repository \cite{github}.

\begin{thebibliography}{XXyyy}
\bibitem[51Kik]{51Kik} Kikuchi (1951)
\bibitem[81Hil]{81Hil} Hillert, Physica (1981)
\bibitem[81Ind]{81Ind} Inden, Physica (1981)
\bibitem[85Hil]{85Hil} Hillert et al, (1985) ionic liquid model
\bibitem[01Hil]{01Hil} Hillert, (2001) CEF
\bibitem[07Luk]{07Luk} Lukas et al (2007)
\bibitem[08Hil]{08Hil} Hillert et al, (2009) corrected quasichemical model
\bibitem[TDB]{TDB} TDB format, see Thermo-Calc guide
\bibitem[14Sun]{14Sun} Sundman et al, to be published in IMMI, (2015)
\bibitem[15Sun]{15Sun} Sundman et al, to be published in Comp.Mat.Sci, (2015)
\bibitem[ocweb]{ocweb} http://www.opencalcalphad.org
\bibitem[github]{github} opencalphad at http://www.github.com
\end{thebibliography}

\end{document}


\begin{eqnarray}
\end{eqnarray}

