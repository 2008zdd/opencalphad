\documentclass[12pt]{article}
\usepackage[latin1]{inputenc}
\topmargin -1mm
\oddsidemargin -1mm
\evensidemargin -1mm
\textwidth 155mm
\textheight 220mm
\parskip 2mm
\parindent 3mm
%\pagestyle{empty}

\begin{document}

\begin{center}

{\large \bf TP function Package (TPF)}

Bo Sundman, \today

\end{center}

This is a very preliminary description of the data structure used in
the utility package TPF written in Fortran08.

The TP function package is used by the General Thermodynamic Pacakage
(GTP) to handle expressions dependent on $T$ and $P$.  It calculates
the value of the function and the first and second derivates thereof.

All real numbers are double precision unless specifically noted.

This documententation is neither complete nor up to date with the most
resent version.  The whole structure (and depending subroutines)
should be revised when all major parts have been implemented and
tested.

\vspace{5mm}

{\bf Documentation updating software}

One very difficult thing with software that is alive and changing is
to update the documention.  So my feeling for documentation is that
once a software is documented it is dead as it is almost impossible to
update source code and documentation in parallell.

To avoid this I have now written a simple documentation update
software that uses a basic documentation written as a LaTeX file
together with the source code to extract verbatim sections.  When
running this update software with the current documentation file and a
new source code it compares the existing verbatim sections with those
it can find in the new source code and flaggs all differences.  This
is an easy method to find what has changed in the source code since
last time I worked on the documentation.

The verbatim sections in the source code is enclosed by\\
\verb/!\begin{verbatim}/\\
\verb/extract from source code /\\
\verb/!\end{verbatim} /\\
These can be inserted anywhere.  Normally they enclose important
sections like data structure definitions and declarations of
subroutines and functions.  New verbatim sections are also flagged as
well as verbatim sections that has disappeared.  The update software
writes a new LaTeX documentation file which require some editing but
at least is correct.

With this software I hope to be able to keep the software documented
also during the development stage.

This document is an example of this software.  I assume there are
already many similar software but this one suits my style of
programming.

\newpage

\tableofcontents

\newpage

\section{TP functions}

TP functions are a subpackage inside GTP.  These can store simple
expressions depending on T and P including unary functions LN and EXP
(and maybe some more).  They can return a function value and the first
and second derivative with respect to T and P.  There are three
datatypes associated with these functions.

\subsection{Data structures}

Several data structures are used inside the TP package and some of
them also in programs dependent on this.

\subsubsection{Structure to store expressions of TP functions}

This is the error code

\begin{verbatim}
 TYPE gtp_parerr
! This record contains the global error code.  In parallell processing each
! parallell processes has its own error code copied to this if nonzero
! it should be replaced by gtperr for separate errors in treads
    INTEGER :: bmperr
 END TYPE gtp_parerr
 TYPE(gtp_parerr) :: gx
\end{verbatim}

\subsubsection{Structure to store expressions of TP functions}

In this record the coefficients and powers of T and P and possible
unary functions used is stored.  There is a simple parser that can
read a simple TP function similar to what TC has.  My idea has been to
extend this a bit by allowing parenthesis in a more flexible way but
that is not yet implemented.  A TP function can refer to another TP
function using the link array.  This means a recursive evaluation of
TP functions must be implemented.

This information must not be changed during parallell processing.
During assessments one may calculate each experimental equilibria in
parallell and as the coefficients of TP functions are varied by the
assessment procedure one may think that this array must also be
separate for each parallell process.  But changes of coefficients are
not done during the parallell execution so there is no problem.

\begin{verbatim}
  TYPE tpfun_expression
! Coefficients, T and P powers, unary functions and links to other functions
     integer noofcoeffs,nextfrex
     double precision, dimension(:), pointer :: coeffs
! each coefficient kan have powers of T and P/V and links to other TPFUNS
! and be multiplied with a following LOG or EXP term.
     integer, dimension(:), pointer :: tpow
     integer, dimension(:), pointer :: ppow
     integer, dimension(:), pointer :: wpow
     integer, dimension(:), pointer :: plevel
     integer, dimension(:), pointer :: link
  END TYPE tpfun_expression
! These records are allocated when needed, not stored in arrays
\end{verbatim}

\subsubsection{Function root record type}

In this record a name of the function is stored and the number of
temperature ranges.  For each range a low temperature limit and a link
to a tpfun\_expression record is stored.  There is also a high
temperature limit.  As the number of ranges can vary the arrays are
allocatable (declared as pointers).  At the same time as a root record
is reserved to store a TP function one also reservs a tpfun\_parres
record and in this the last calculated result of the function is
stored.  Saving the last calculated values speeds up calculation
because the same function is often used in many parameters and needed
many times for the same values of T and P.  But in parallell
processing the values of T and P can be different in each processor
and thus the results mut be separate in each.

\begin{verbatim}
  TYPE tpfun_root
! Root of a TP function including name with links to coefficients and codes
! and results.  Note that during calculations which can be parallellized
! the results can be different for each parallell process
     character*(lenfnsym) symbol
! limits is the low temperature limit for each range
! funlinks links to expression records for each range
! each range can have its own function, status indicate if T and P or T and V
     integer noofranges,nextfree,status
     double precision, dimension(:), pointer :: limits
     TYPE(tpfun_expression), dimension(:), pointer :: funlinks
     double precision hightlimit
  END TYPE tpfun_root
! These records are stored in arrays as the actual function is global but each
! equilibrium has its own result array (tpfun_parres) depending on the local
! values of T and P/V.  The same indiex is used in the global and local arrays.
! allocated in init_gtp
  TYPE(tpfun_root), private, dimension(:), pointer :: tpfuns
\end{verbatim}

\subsubsection{Structure for calculated results of TP functions}

This record contains the last values of T and P used for calculation
the TP function and the results including first and second derivatives
of T and P.  These are ordered as F, F.T, F.P, F.T.T, F.T.P and F.P.P.
The reason to have this as a separate array is that in parallell
processing the values of T and P may be different in each process and
thus the results also, whereas the values of the coefficients are the
same.  The size of all the function records are allocated dynamically
by the tpfun\_init subroutine.

\begin{verbatim}
  TYPE tpfun_parres
! Contains a TP results, 6 double for results and 2 doubles for T and P
! values used to calculate the results
! Note that during calculations which can be parallellized the final
! results can be different for each parallell process
     double precision, dimension(2) :: tpused
     double precision, dimension(6) :: results
  END TYPE tpfun_parres
! This array is local to the gtp_equilibrium_data record
\end{verbatim}

\subsection{Subroutines}

\subsubsection{Initialization}

\begin{verbatim}
 SUBROUTINE tpfun_init(nf,tpres)
! allocate tpfuns and create a free list inside the tpfuns
   implicit double precision (a-h,o-z)
! use tpres declared externally for parallel processing
  TYPE(tpfun_parres), dimension(:), pointer :: tpres
\end{verbatim}

\subsubsection{Number of function}

\begin{verbatim}
 integer function notpf()
! number of tpfunctions because freetpfun is private
\end{verbatim}

\subsubsection{Find function}

\begin{verbatim}
  SUBROUTINE find_tpfun_by_name(name,lrot)
! returns the location of a TP function
    character name*(*),name1*16
\end{verbatim}

\subsubsection{Evaluate function}

\begin{verbatim}
 subroutine eval_tpfun(lrot,tpval,result,tpres)
!    subroutine eval_tpfun(lrot,tpval,symval,result)
! evaluate a TP function with several T ranges
   implicit double precision (a-h,o-z)
   dimension tpval(2),result(6)
   TYPE(tpfun_parres), dimension(:), pointer :: tpres
\end{verbatim}

\subsubsection{List function}

\begin{verbatim}
 subroutine list_tpfun(lrot,nosym,str)
! lists a TP function with several ranges into string str
   implicit double precision (a-h,o-z)
   character str*(*)
   integer nosym
\end{verbatim}

\subsubsection{List all functions}

\begin{verbatim}
 subroutine list_all_funs(lut)
! list all functions except those starting with _ (parameters)
\end{verbatim}

\subsubsection{Compile function expression}

\begin{verbatim}
 SUBROUTINE ct1xfn(string,ip,nc,coeff,koder)
!...compiles an expression in string from position ip
!     it can refer to T and P or symbols in fnsym
!     compiled expression returned in coeff and koder
!
! >>> this is very messy
!
!...algorithm for function extraction
! 10*T**2 -5*T*LOG(T) +4*EXP(-5*T**(-1))
!
! AT LABEL 100 start of expression or after (
! sign=1
! -, sign=-1                              goto 200
! +, skip
!
! AT LABEL 200 after sign
! if A-Z                                  goto 300
! if 0-9, extract number                  goto 400
! (                                       goto 100
! ;                                       END or ERROR
! empty                                   END or ERROR
! anything else                           ERROR
!
! AT LABEL 300 symbol
! if T or P, extract power if any incl () goto 400
! unary fkn? extract (                    goto 100
! symbol                                  goto 400
!
! AT LABEL 400 after factor
! -, sign=-1                              goto 200
! sign=1
! +, skip                                 goto 200
! )                                       goto 400
! ** or ^ extract and store power incl () goto 400
! *                                       goto 200
! empty                                   goto 900
!
! for TDB compatibility skip #
! 900 END
! check consistency
\end{verbatim}

\subsubsection{Extract a symbol while compiling}

\begin{verbatim}
 subroutine ct1getsym(string,ip,symbol)
!...extracts an symbol
   implicit double precision (a-h,o-z)
   PARAMETER (lenfnsym=16)
   character string*(*),ch1*1,chs*1,symbol*(*),localsym*(lenfnsym)
\end{verbatim}

\subsubsection{Extract an integer power while compiling}

\begin{verbatim}
 subroutine ct1power(string,ip,ipower)
!...extracts an integer power possibly surrounded by ( )
\end{verbatim}

\subsubsection{Create a root record for the function}

\begin{verbatim}
 subroutine ct1mfn(symbol,nranges,tlimits,lokexpr,lrot)
!...creates a root record with name symbol and temperature ranges
! highest T limit is in tlimits(nranges+1)
   implicit double precision (a-h,o-z)
   character*(*) symbol,name*16
   TYPE(tpfun_expression), dimension(*) :: lokexpr
   real tlimits(*)
\end{verbatim}

\subsubsection{Create a data structure of the function}

\begin{verbatim}
 subroutine ct1mexpr(nc,coeff,koder,lrot)
!...makes a datastructure of an expression. root is returned in lrot
   implicit double precision (a-h,o-z)
   TYPE(tpfun_expression), pointer :: lrot
   TYPE(tpfun_expression), pointer :: noexpr
   dimension coeff(*),koder(5,*)
\end{verbatim}

\subsubsection{Create a data structure of the function}

\begin{verbatim}
 subroutine ct1efn(inrot,tpval,val,tpres)
!...evaluates a datastructure of an expression. Value returned in val
!     inrot is root expression tpfunction record
!     tpval is valuse of T and P, symval is values of symbols
! first and second derivatives of T and P also calculated and returned
! in order F, F.T, F.P, F.T.T, F.T.P, F.P.P
!
! if function already calculated one should never enter this subroutine
!
! It can call "itself" by reference to another TP function and for
! that case one must store results in levels.
\end{verbatim}

\subsubsection{Write an expression to a character}

\begin{verbatim}
 subroutine ct1wfn(exprot,tps,string,ip)
!...writes an expression into string starting at ip
!     lrot is an index to an tpexpr record
   implicit double precision (a-h,o-z)
   parameter (levl=5,nunary=4)
   TYPE(tpfun_expression), pointer :: exprot
   PARAMETER (lenfnsym=16)
   dimension coeff(levl),koder(5,levl)
   character tps(2)*(*),ch1*1,cht*1
   character string*(*),extsym*(lenfnsym),unary(nunary)*4
   DATA unary/'LOG ','LN  ','EXP ','ERF '/
\end{verbatim}

\subsubsection{Write an power to a character}

\begin{verbatim}
 subroutine ct1wpow(string,ip,tps,mult,npow)
!...writes "ips" with a power if needed and a * before or after
   implicit double precision (a-h,o-z)
   character string*(*),tps*(*)
\end{verbatim}

\subsubsection{Interactivite input of a function}

\begin{verbatim}
 subroutine enter_tpfun_interactivly(cline,ip,longline,jp)
! interactive input of a TP expression
   implicit double precision (a-h,o-z)
   character cline*(*),longline*(*),line*80,ch1*1
\end{verbatim}

\subsubsection{Deallocate a function}

\begin{verbatim}
 subroutine tpfun_deallocate
! deallocates all arrays associated with a TP function
\end{verbatim}

\subsubsection{Enter an expression}

\begin{verbatim}
 subroutine enter_tpfun(symbol,text,lrot)
! creates a data structure for a TP function called symbol with several ranges
! text is whole expression
! default low temperature limit is 298.16; high 6000
   implicit double precision (a-h,o-z)
! max number of ranges, max number of coefficents in each range
   parameter (mrange=20,mc=15)
   real tlim(mrange)
   dimension coeff(mc),koder(5,mc)
   TYPE(tpfun_expression) :: links(mrange)
   TYPE(tpfun_expression), pointer :: ltpexpr
   character*(*) text,symbol
   character ch1*1
! check if function already entered, there are freetpfun-1 of them
! ignore functions that start with a "_" as they are parameters
\end{verbatim}

\subsubsection{Called when calculated a nested TP function}

\begin{verbatim}
 subroutine nested_tpfun(lrot,tpval,nyrot)
! called from ct1efn when a it calls another TP function that must be
! evaluated.  nyrot is the link to the ct1efn in the correct range
   implicit double precision (a-h,o-z)
   dimension tpval(2)
   TYPE(tpfun_expression), pointer :: nyrot
! use lowest range for all T values lower than first upper limit
! and highest range for all T values higher than the next highest limit
! one should signal if T is lower than lowest limit or higher than highest
! used  saved reults if same T and P
\end{verbatim}

\subsubsection{Compare names allowing abbreviations}

\begin{verbatim}
 logical function compare_abbrev(name1,name2)
! returns TRUE if name1 is an abbreviation of name2
! termintaes when a space is found in name1
! each part between _ or - can be abbreviated from the left
! case insensitive. Only 36 first characters compared
\end{verbatim}

\subsubsection{Called when writing an expression to a file}

\begin{verbatim}
 subroutine tpfunlim(i1,i2)
! used when saving TP funs to files in GTP as highexpr and freexpr are private
! and they are saved at several places in the file (but that can be changed)
\end{verbatim}

\subsubsection{Write TP functions to a file}

\begin{verbatim}
 subroutine tpfunsave(lut,form)
! save tpfundata on a file, unfinished
   implicit double precision (a-h,o-z)
   TYPE(tpfun_expression), pointer :: exprot
\end{verbatim}

\subsubsection{Write a TP function to a file}

\begin{verbatim}
 subroutine save1tpfun(lut,form,jfun)
! save one tpfun (a parameter) with index jfun on a file
\end{verbatim}

\subsubsection{Read a TP function from a file}

\begin{verbatim}
 subroutine read1tpfun(lut,jfun)
! read one unformatted tpfun (a parameter) with index jfun
\end{verbatim}

\subsubsection{Read all TP functions from a file}

\begin{verbatim}
 subroutine tpfunread(lut)
! read tpfundata from a file
\end{verbatim}

\section{Conclusions}

Thats all!

\begin{thebibliography}{77Zzz}
\bibitem[07Luk]{07Luk} Lukas, Fries and Sundman
\end{thebibliography}

\end{document}
\end{document}
\end{document}
\end{document}
