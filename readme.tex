\documentclass[12pt]{article}
\usepackage[latin1]{inputenc}
\usepackage{graphicx,subfigure}
\topmargin -1mm
\oddsidemargin -1mm
\evensidemargin -1mm
\textwidth 165mm
\textheight 220mm
\parskip 2mm
\parindent 3mm
%\pagestyle{empty}

\begin{document}
\begin{center}
{\Large \bf Some important facts about this release of the
  OC software, version 2.}

Bo Sundman,  7 January 2015

bo.sundman@gmail.com

http://www.opencalphad.org

opencalphad repository at https://www.github.com

\end{center}

This software is free and under development.  Your feedback about
problems and errors is important to make it better.

OC cannot replace your favourite thermodynamic software today or
tomorrow but the main advantage is that you have access to the source
code and can (with some efforts) add or fix things yourself that you
are missing in your favourite software.

\section{Getting started}

\begin{itemize}
\item The code is written in the new Fortran standard and requires a
  compiler like GNU fortran 4.8 or similar.

\item On windows copy the linkmake.txt file to a file linkmake.cmd and
  it will compile and link the code.

\item The make command will compile and generate the executable file
  from the source code.  If you are using Windows and have not
  installed the make utility the file ``linkmake.txt'' can be used
  coped to a cmd file to compile and link the code.

\item There is a oc-news.pdf file with the new things in this version
  relative to version 1.

\item The present documentation of the source code is in the directory
  ``documentationupdate''.  

  A very rudimentary user guide is in the file ``ochelp.tex''.  You
  can make a PDF of this using pdflatex.  

  If you make a copy of ochelp.tex to ochelp.hlp in the same directory
  as the program it can also be used for on line help typing ?

  The ``macro'' directory has examples for a variety of calculations.

\item The source code is also available on a repository at Github,
  https://www.github.com, with the name opencalphad.  Contributions of
  new and improved source code are welcome.  Contact Bo Sundman if you
  want to know things to do.

\item There is a wiki at the opencalphad repository, use this for
  information, questions and discussions on the software.

\item The command line interface has a ``VAX/VMS'' flavour which
  reflects the age of the developer.  It means the commands are
  ``verbs'' like {\em set, list, calculate, enter etc.}.  After the
  verb several objects are usually possible.  There is some redundancy
  so the same effect can be achieved by different combinations of
  verbs and objects.

  If you are burning to develop another user interface please do not
  hesitate.

\item Thermodynamic data can be read from a (unencrypted) TDB file
  {\em read tdb ``filename''} or entered interactively.  Beware, there
  is no safe way to save data you have entered interactively except in
  a log file {\em set log ``filename''} and then use the log file as a
  macro.

  In this version you can select the elements from the tdb file, but
  all phases with at least one of the selected elements are included.
  You can use the SET PHASE phase-name STATUS SUSPEND to remove phases
  you do not want.

\item You can save an UNFORMATTED file with the default equilibrium
  read it back as UNFORMATTED but this is fragile and the file
  structure will probably be changed several times before it is
  reliable so the file may not be used in a later version.  There is
  no way to save results from STEP or MAP commands on a file.

\item Setting conditions is very similar to the Thermo-Calc software.
  The command is {\em set cond ``state variable'' = ``value''}.  The
  safest conditions for calculating an equilibrium, i.e. which has
  most chance to converge, is to set values on T, P and N(element),
  i.e. the total amount of each element.  The table at the end gives a
  list of state variable symbols and their meaning.

  The intention is that you should be able to combine any set of
  conditions to calculate the equilibrium, i.e. you can combine
  conditions on mole fractions, mass fractions, fix phases, chemical
  potentials etc.  You can also prescribe phases as stable (fix).  In
  a future release one will be able to give expressions of conditions,
  not just a value of a single state variable.

  Conditions on V, H, S etc are not yet implemented.

\item The command {\em calculate equilibrium (c e)} tries to calculate
  the equilibrium.  As the minimizer needs a guess of stable phases
  and their start constitution it always tries first to invoke a
  global grid minimizer.  If you want to provide a guess of the set of
  stable phases yourself use the command {\em calculate no\_global (c
    n)}.  To set an initial stable phase use the command {\em set
    phase const}.

\item The grid minimizer that calculates start points for the general
  minimizer is very primitive.  If you have ideas how to improve it
  you are welcome to provide advice or code.

\item You will frequently have problems with convergence, many times
  there will be error messages and sometimes the software may converge
  to the wrong equilibrium.  In this beta test version you should
  always check the result with you favourite thermodynamic software.

  If the calculation does not converge try to use the command {\em ``c
    n''} two or three times to continue to iterate from the set of
  phases you have.  The {\em ``c e''} command starts the grid
  minimiser and, therefore, will give the same result each time.

  You can also try to increase the number of iterations {\em set num
    500 ,,,,}.  Or you can manually set the phases you think are
  stable {\em set phase const} followed by {\em ``c n''}.  Or you can
  try to simplify the conditions for a first calculation and then
  change them to those you are interested in and for each change
  calculate using {\em ``c n''}.  Calculations at temperatures and
  compositions where the system is single phase have a higher chance
  of success.  The algorithm to change the set of stable phases is
  fragile and has not been fine tuned.

\item In version 2 the step and map procedures are implemented but
  full if bugs.  It uses GNUPLOT to handle the graphics.  

\item The ``dot derivative'' facility has been implemented for a
  limited set of cases of derivatives with respect to $T$ and $P$,
  like ``H,T'' for the heat capacity.  

\item The partitioning of phases with an ordered and disordered part
  has been implemented slightly simpler than in Thermo-Calc.  See the
  documentation of the model package.

\item The code is very messy, I had hoped to have time to clean it up
  before this release but that will be for the next one.

\item The big thing to do now is to develop an assessment module.
  Anyone with ideas are welcome to contribute.

\item To report errors and problems please attach a macro file that
  reproduces the problem.  To create a macro file run the command
  interface with a log file and edit the log file to become a macro.
  Both the macro file and the model parameters (preferably in a tdb
  file) must be submitted.  Try to find the simplest case that
  reproduces the error.

\end{itemize}

\section{A summary of state variables.}

The state variables in the user interface have their common symbols,
$T$ for temperature, $P$ for pressure, $N$ for the total amount of
moles, ``$N$(element)'' for the amount of moles of a component,
``$X$(element)'' for the mole fraction ``$MU$(element)'' for the
chemical potential, ``$AC$(element)'' for the activity.  The symbol
$B$ is used for the total mass (copied from the Thermo-Calc software),
``$B$(element)'' for the mass of an element and ``$W$(element)'' for
the mass fraction.  There are many more state variables like $H$, $G$
etc, see the table, but not all of them can be used as conditions.

\begin{table}
\caption{A very preliminary table with the state variables and their
internal representation.  Some model parameter properties are also
included.}\label{tab:statevar}
\begin{tabular}{|llccll|}\hline
Symbol & Id & \multicolumn{2}{c}{Index} & Normalizing & Meaning\\
       &    & 1 & 2                     &  suffix     & \\\hline
\multicolumn{6}{|c|}{Intensive properties}\\\hline
T      & 1  & -         & -    & - & Temperature\\
P      & 2  & -         & -    & - & Pressure\\
MU     & 3  & component & -/phase  & - & Chemical potential\\
AC     & 4  & component & -/phase  & - & Activity\\
LNAC   & 5  & component & -/phase  & - & LN(activity)\\\hline
\multicolumn{6}{|c|}{Extensive properties}\\\hline
U      & 10 & -/phase\#set & - & - & Internal energy for system\\
UM     & 11 & -/phase\#set & - & M & Internal energy per mole\\
UW     & 12 & -/phase\#set & - & W & Internal energy per mass\\
UV     & 13 & -/phase\#set & - & V & Internal energy per m$^3$\\
UF     & 14 & phase\#set   & - & F & Internal energy per formula unit\\
Sz     & 2z & -/phase\#set & - & - & entropy\\
Vz     & 3z & -/phase\#set & - & - & volume\\
Hz     & 4z & -/phase\#set & - & - & enthalpy\\
Az     & 5z & -/phase\#set & - & - & Helmholtz energy\\
Gz     & 6z & -/phase\#set & - & - & Gibbs energy\\
NPz    & 7z &  phase\#set & - & - & Moles of phase\\
BPz    & 8z & phase\#set & - & - & Mass of phase\\
Qz     & 9z & phase\#set & - & -  & Stability of phase\\
DGz    & 10z & phase\#set & - & -  & Driving force of phase\\
Nz     & 11z & -/phase\#set/comp & -/comp & -  & Moles of component\\
X      & 111 & phase\#set/comp & -/comp & 0  & Mole fraction\\
X\%    & 111 & phase\#set/comp & -/comp & 100 & Mole per cent\\
Bz     & 12z & -/phase\#set/comp & -/comp & -  & Mass of component\\
W      & 122 & phase\#set/comp & -/comp & 0 & Mass per cent\\
W\%    & 122 & phase\#set/comp & -/comp & 100 & Mass per cent\\
Y      & 130 & phase\#set & const\#subl & -& Constituent fraction\\\hline
\multicolumn{6}{|c|}{Some model parameter identifiers}\\\hline
TC     & - & phase\#set & - & - & Curie temperature\\
BMAG   & - & phase\#set & - & - & Aver. Bohr magneton number\\
MQ\&   & - & phase\#set & constituent & - & Mobility\\
THET   & - & phase\#set & - & - & Debye temperature\\\hline
\end{tabular}
\end{table}

You can specify that a phases should be stable by the command {\em set
status phase}.  For example to calculate the melting point of an alloy
after specifying the composition and making a calculation at fixed T
and P, you can give the commands {\em set cond T=none; set status
liquid=fix 0; c n}.  (The commands must be given on separate lines).

In some cases it is convenient to use the command {\em set input
amount} to specify the amounts of a redundant set of species that are
added together to make up the overall composition.

\section{Manipulating the source code}

The OC software is provided with a GNU license which means that you
have the source code and can use it and modify it as you wish as long
as you do not try to make money of it.  If you want to include the OC
software (or part of it) in a commercial program you must contact the
copyright owners to have another license.

There is a fairly extensive documentation of the source code in the
directory ``documentationupdate'' and if you look at the code itself
there are some comments there too.  I have spent a lot of effort to
make the datastructures general and flexible to handle multicomponent
and multiphase systems.  But there was quite a lot of redundancy
introduced during the development that eventually will be removed.
The set of subroutines is less structured and one problem has been
that this code was my first attempt to use the new Fortran standard so
there are probably many things that can be made simpler.  You are
welcome to point out where this can be done.  The hope is to release
this code before the end of 2014.

As I have understood the data structure (TYPE) in the new Fortran
standard is more or less identical to the structures used in C++ so it
should not be too difficult to combine code written in these
languages.

\section{Application software library}

There are some routines provided that makes it possible to integrate
the OC software in application programs.  Those can be found in the
directory ``TQlib'' together with some sample test program.  The
source library is liboctq.F90.  To make an object file of this it must
be compiled together with the liboceq.mod file.  To link the test
programs you must also use the liboceq.a library.

Use only subroutines in this library to access the OC software, do not
call directly subroutines inside the OC code as they may not be
available or have different functionallity in a future release.  If
you miss some routines please put a request at the GITHUB wiki.

\section{Macro examples}

To help you get started calculating a number of examples is provided
in the ``macro'' directory.  There is a separate set of macros for
version 1 that you may also be interested in, you find them with the
source code for version 1.  Note that the extention of macro file is
now OCM, with version 1 it was BMM.

The examples for version 2 mainly concern mapping and stepping.  They
show that there are still issues to solve.

\begin{enumerate}
\item Mapping the binary phase diagram for Ag-Cu and setting ranges
  for plotting.

\item Mapping the Cr-Mo binary diagram with a miscibility gap.  The
  curves for the miscibility gap does not close.

\item Mapping the binary C-Fe phase diagram.  Does not extend to pure Fe.

\item Mapping the binary U-O phase diagram.  Several start points needed.

\item Mapping the binary Fe-Mo phase diagram.  Two start points
  needed.

\item Mapping the isopleth for CrFeMo.  Missing lines at low T.

\item Mapping a high speed steel.  Many lines are missing or wrong.

\item Mapping the low temperature ordering in Fe-Ni.

\item Mapping the DFT calculated Re-W phase diagram (without liquid).
  Also some property diagrams for varying composition.

\item Property (step) diagram for a high speed steel vs T.  Plotting
  phase amount, mass percent of Cr in various phases, enthalpy and
  heat capacity.  The last an example of a dot derivative.

\item Stepping G curves for Ag-Cu vs composition.

\item Stepping an ideal gas with H and O vs T plotting speciation,
enthalpy and heat capacity.

\item Stepping property diagram for low temperature Fe-Ni system

\item Calculate and plot the heat capacity for the FeNi3 ordered phase.

\item Step the G curves for Fe-Mo at 1400 K.

\item Calculate varions properties for a SAF2507 duplex stainless steel.

\end{enumerate}

{\large \bf Have fun and make OC better!}

\end{document}

