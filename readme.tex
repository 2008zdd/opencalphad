\documentclass[12pt]{article}
\usepackage[latin1]{inputenc}
\usepackage{graphicx,subfigure}
\topmargin -1mm
\oddsidemargin -1mm
\evensidemargin -1mm
\textwidth 165mm
\textheight 220mm
\parskip 2mm
\parindent 3mm
%\pagestyle{empty}

\begin{document}
\begin{center}
{\Large \bf Some important facts about this beta test release of the
OC software.}

Bo Sundman, \today

http://www.opencalphad.org

bo.sundman@gmail.com
\end{center}

{\large \bf Please be aware that this software is a beta test version.
Your feedback about problems and errors is important to make it
better.}

OC cannot replace your favourite thermodynamic software today or
tomorrow but the main advantage is that you have access to the source
code and can (with some efforts) add or fix things yourself that you
are missing in your favourite software.

\section{Getting started}

\begin{itemize}
\item The code is written in the new Fortran standard and requires 
a compiler like GNU fortran 4.4 or similar.

\item The make command will compile and generate the executable file
from the source code.  If you are using Windows and have not installed
the make utility the file ``linkmake.txt'' can be used coped to a cmd
file to compile and link the code.

\item The documentation of the source code is in the directory
``documentationupdate''.  A very rudimentary user guide (also
available on line in the file ``ochelp.hlp'') is the ``manual''directory).
The ``macro'' directory has examples for a variety of calculations.

\item There is a website, http:://www.opencalphad.com, for
information, questions and discussions on the software and databases.

\item Contributions of new and improved source code are welcome.
Contact Bo Sundman if you want to know how.

\item Contributions of new source code are welcome.  Contact Bo
Sundman if you want to know how.

\item The command line interface has a ``VAX/VMS'' flavour which reflects the
age of the developer.  It means the commands are ``verbs'' like {\em
set, list, calculate, enter etc.}.  After the verb several objects are
usually possible.  There is some redundancy so the same effect can be
achieved by different combinations of verbs and objects.

If you are burning to develop another user interface you are welcome
to do so.

\item Thermodynamic data can be read from a (unencrypted) TDB file
{\em read tdb ``filename''} or entered interactively.  Beware, there is
no way to save data you have entered interactively except in a log file
{\em set log ``filename''} and then use the log file as a macro.

In this version you cannot select the elements from the tdb file, all
elements are entered.  You must edit the tdb file in advance to select
the elements (normally it is enough to remove the elements you do not
want in the beginning of the file).  In the next release it will be
possible to select elements and phases from the tdb file also inside
the OC software.

\item Setting conditions is very similar to the Thermo-Calc software.
The command is {\em set cond ``state variable'' = ``value''}.  The
safest conditions for calculating an equilibrium, i.e. which has most
chance to converge, is to set values on T, P and N(element), i.e. the
total amount of each element.  The table at the end gives a list of
state variable symbols and their meaning.

The intention is that you should be able to combine any set of
conditions to calculate the equilibrium, i.e. you can combine
conditions on mole fractions, mass fractions, fix phases, chemical
potentials etc.  In the next release one will be able to give
expressions of conditions, not just a single state variable.

\item The command {\em calculate equilibrium (c e)} tries to calculate
the equilibrium.  As the minimizer needs a guess of stable phases and
their start constitution it always tries first to invoke a global grid
minimizer.  If you want to provide a guess of the set of stable phases
yourself use the command {\em calculate no\_global (c n)}.  To set an
initial stable phase use the command {\em set phase const}.

\item The grid minimizer that calculates start points for the general
minimizer is very primitive.  If you have ideas how to improve it you
are welcome to provide advice or code.

\item You will frequently have problems with convergence, many times
there will be error messages and sometimes the software may converge
to the wrong equilibrium.  In this beta test version you should always
check the result with you favourite thermodynamic software.

If the calculation does not converge try to use the command {\em ``c
n''} two or three times to continue to iterate from the set of phases
you have.  The {\em ``c e''} command starts the grid minimiser and,
therefore, will give the same result each time.  

You can also try to increase the number of iterations {\em set num 500
,,,,}.  Or you can manually set the phases you think are stable {\em
set phase const} followed by {\em ``c n''}.  Or you can try to
simplify the conditions for a first calculation and then change them
to those you are interested in and for each change calculate using
{\em ``c n''}.  Calculations at temperatures and compositions where
the system is single phase have a higher chance of success.  The
algorithm to change the set of stable phases is fragile and has not
been fine tuned.

\item There is a simple step procedure to calculate property diagrams.
It uses gnuplot to handle the graphics.  If you are interested in this
feature there will be a discussion group on the future design of the
step and map (of phase diagrams) commands and graphical output.

\item To report errors and problems please attach a macro file that
reproduces the problem.  To create a macro file run the command
interface with a log file and edit the log file to be a macro.  Both
the macro file and the model parameters (preferably in a tdb file)
must be submitted.  Try to find the simplest case that reproduces the
error.
\end{itemize}

\section{A summary of state variables.}

The state variables in the user interface have their common symbols,
$T$ for temperature, $P$ for pressure, $N$ for the total amount of
moles, ``$N$(element)'' for the amount of moles of a component,
``$X$(element)'' for the mole fraction ``$MU$(element)'' for the
chemical potential, ``$AC$(element)'' for the activity.  The symbol
$B$ is used for the total mass (copied from the Thermo-Calc software),
``$B$(element)'' for the mass of an element and ``$W$(element)'' for
the mass fraction.  There are many more state variables like $H$, $G$
etc, see the table, but not all of them can be used as conditions.

\begin{table}
\caption{A very preliminary table with the state variables and their
internal representation.  Some model parameter properties are also
included.}\label{tab:statevar}
\begin{tabular}{|llccll|}\hline
Symbol & Id & \multicolumn{2}{c}{Index} & Normalizing & Meaning\\
       &    & 1 & 2                     &  suffix     & \\\hline
\multicolumn{6}{|c|}{Intensive properties}\\\hline
T      & 1  & -         & -    & - & Temperature\\
P      & 2  & -         & -    & - & Pressure\\
MU     & 3  & component & -/phase  & - & Chemical potential\\
AC     & 4  & component & -/phase  & - & Activity\\
LNAC   & 5  & component & -/phase  & - & LN(activity)\\\hline
\multicolumn{6}{|c|}{Extensive properties}\\\hline
U      & 10 & -/phase\#set & - & - & Internal energy for system\\
UM     & 11 & -/phase\#set & - & M & Internal energy per mole\\
UW     & 12 & -/phase\#set & - & W & Internal energy per mass\\
UV     & 13 & -/phase\#set & - & V & Internal energy per m$^3$\\
UF     & 14 & phase\#set   & - & F & Internal energy per formula unit\\
Sz     & 2z & -/phase\#set & - & - & entropy\\
Vz     & 3z & -/phase\#set & - & - & volume\\
Hz     & 4z & -/phase\#set & - & - & enthalpy\\
Az     & 5z & -/phase\#set & - & - & Helmholtz energy\\
Gz     & 6z & -/phase\#set & - & - & Gibbs energy\\
NPz    & 7z &  phase\#set & - & - & Moles of phase\\
BPz    & 8z & phase\#set & - & - & Mass of phase\\
Qz     & 9z & phase\#set & - & -  & Stability of phase\\
DGz    & 10z & phase\#set & - & -  & Driving force of phase\\
Nz     & 11z & -/phase\#set/comp & -/comp & -  & Moles of component\\
X      & 111 & phase\#set/comp & -/comp & 0  & Mole fraction\\
X\%    & 111 & phase\#set/comp & -/comp & 100 & Mole per cent\\
Bz     & 12z & -/phase\#set/comp & -/comp & -  & Mass of component\\
W      & 122 & phase\#set/comp & -/comp & 0 & Mass per cent\\
W\%    & 122 & phase\#set/comp & -/comp & 100 & Mass per cent\\
Y      & 130 & phase\#set & const\#subl & -& Constituent fraction\\\hline
\multicolumn{6}{|c|}{Some model parameter identifiers}\\\hline
TC     & - & phase\#set & - & - & Curie temperature\\
BMAG   & - & phase\#set & - & - & Aver. Bohr magneton number\\
MQ\&   & - & phase\#set & constituent & - & Mobility\\
THET   & - & phase\#set & - & - & Debye temperature\\\hline
\end{tabular}
\end{table}

You can specify that a phases should be stable by the command {\em set
status phase}.  For example to calculate the melting point of an alloy
after specifying the composition and making a calculation at fixed T
and P, you can give the commands {\em set cond T=none; set status
liquid=fix 0; c n}.  (The commands must be given on separate lines).

In some cases it is convenient to use the command {\em set input
amount} to specify the amounts of a redundant set of species that are
added together to make up the overall composition.

\section{Manipulating the source code}

The OC software is provided with a GNU license which means that you
have the source code and can use it and modify it as you wish as long
as you do not try to make money of it.  If you want to include the OC
software in a commercial program you must contact the copyright
owners.

There is a fairly extensive documentation of the source code in the
directory ``documentationupdate'' and if you look at the code itself
there are some comments there too.  I have spent a lot of effort to
make the datastructures general and flexible to handle multicomponent
and multiphase systems.  But there was quite a lot of redundancy
introduced during the development that eventually will be removed.
The set of subroutines is less structured and one problem has been
that this code was my first attempt to use the new Fortran standard so
there are probably many things that can be made simpler.  You are
welcome to point out where this can be done.  The hope is to have a
new release before the end of 2013 with an improved version of this
part of the code and some more thermodynamic models and facilities for
conditions and generating diagrams will be available.

As I have understood the data structures (TYPE) in the new Fortran
standard is more or less identical to those used in C++ so it should
not be too difficult to combine code written in these languages.

\section{Application software library}

There are some routines provided that makes it possible to integrate
the OC software in application programs.  Those can be found in the
directory ``TQlib'' together with a sample test program.  To compile
and run this the OClib.a file and some of the ``mod'' files must be
copied to this directory.

Use only subroutines in this library to access the OC software, do not
call directly subroutines inside the OC code as they may not be
available or have different functionallity in a future release.  If
you miss some routines please contact the OC software group.

\section{Examples}

To help you get started calculating a number of examples is
provided in the ``macro'' directory.

\begin{enumerate}
\item Entering data manually for an ideallized system (Al-Ni) with
ordering, setting ordering options and conditions and calculating an
equilibrium, ocex01A.

\item Combining reading a TDB file and entering data manually in a
ternary system (Al-Cr-Ni), with fcc and bcc ordering.  Setting
ordering options and conditions and calculating some equilibria,
ocex01B.

\item Entering some parameters for various physical models for a
fictive binary system.  Checking how they vary with composition,
ocex01C.

\item Reading a TDB file for a binary system (Cr-Fe), setting
conditions and calculating several equilibria for different
conditions, ocex02A.

\item Reading a TDB file for a ternary system (C-Cr-Fe), and calculate
several equilibrium for different conditions, ocex02B.

\item Reading a TDB file for a binary gas system (H-O), setting 
various conditions and calculating equilibria, ocex02C.

\item Reading a TDB file for a multcomponent steel, setting conditions
and calculating an equilibrium.  Setting an axis, stepping and
plotting some curves, ocex03A.
\end{enumerate}

{\large \bf Have fun and make OC grow!}

\end{document}

