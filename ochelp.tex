\documentclass[12pt]{article}
\usepackage[latin1]{inputenc}
\topmargin -1mm
\oddsidemargin -1mm
\evensidemargin -1mm
\textwidth 155mm
\textheight 220mm
\parskip 2mm
\parindent 3mm
\setcounter{secnumdepth}{5}
%\pagestyle{empty}

% look for this to continue update
%%%%%%%%%%%%%%%%%%%%%%%%%%%%%%%% current place
%
% This is a file used for a printable PDF version of the user guide
% AND as on-line help, either directly or processed to remove LaTeX specials
%
%---------------------------------------------------
%
% The first version of this is generated manually but eventially a
% software program should be developed to update this automatically whenever
% the software is changed
%
%---------------------------------------------------
%
% Some advice:
%
% The commands and subcommands are arranged alphabetically
%
% It will be difficult to update the help text for the
% questions after the commands and subcommands as they are normally
% not part of the command monitor.
%
% The _ used in many commands must be replaced by \_ or just -
%
%---------------------------------------------------
%
% The on-line help software will react if the user types a ?
% as answer to a question.  It will search for the help text starting by
% the main command, any subcommand and finally question texts.
% The quesntion text may be difficult to update as already mentioned. 
% Any text found in this file between the start of a question and the next will
% be written on the screen and then the question will be asked again.
%
%---------------------------------------------------
%
\begin{document}

\begin{center}

{\Huge \bf User Guide to the 

Open Calphad software package

version 2.0

}

VERY PRELIMINARY

Bo Sundman, \today

\end{center}

This is a very preliminary version for version 2

\section{Introduction}

The Open Calphad software project aims to provide a hig quality
software for thermodynamic calculations for inorganic systems i.e.
gases. liquid, alloys with may different crystalline phases.

It also provides a framwork to store many different composition
dependent properties of materials.

\section{Some general features}

The command monitor has a menu of command and each of these usually
has submenus and finally some questions may be asked like phase names,
a value or an expression.  At any level the user should be able to
type a ? and get some help, usually an extract from this manual, a
menue or possible answers.

\subsection{Names and symbols}

There are many symbols and names used in this package.  A symbol or
name MUST start with a letter A-Z.  It usually can contain digits and
the underscore character after the intitial letter.  Some special
symbols are also used:

\begin{itemize}
\item /- is used to denote the electron. /+ can be used for a psoitiv charge.
\item \# are used to identify composition sets after a phase name or
sublattice after a constituent name.
\item \& are used in some parameter identifiers to specify the
  constituent for the parameter, mobilites.
\end{itemize}

\subsection{Parameters}

All data is organized relative to a phase and the phase is identified
by a name.  Each phase can have a different model for the composition
dependence but the way to enter model parameters is the same for all
models.  However, the meaning of a model parameter may depend on the
model of the phase.

Many types of data can be stored as explained in the section on
parameter identifiers.  The parameter also has a constituent
specification explained in the constituent array section and possibly
a degree, the meaning of which is model dependent.

The basic syntax of a parameter is

``identifier'' ( ``phase name'' , ``constituent array'' ; ``degree'' ) ``expression'' ``reference''

These parts will now be explained in more detail.

\subsubsection{Parameter Identifiers}

The OC thermodynamic pacakage can handle any property that depend on
composition using the composition models implemented.  It is easy to
extend the number of properties by declaring property identifiers in
ths source code.  The value of such identifiers can be obtained by the
command ``list symbol''.  If the parameters should have an influence
on the Gibbs energy (like the Curie temperature) or a diffusion
coefficient (like the mobility) the necessary code to calculate this
must be added also.

The list here is tentative.  Insensitive to case.

\begin{itemize}
\item G, the Gibbs energy or an interaction parameter.
\item TC, the critical temperature for ferro or antiferro magnetic
ordering using the Inden model.
\item BMAGN, the avarage Bohr magneton number using the Inden model.
\item CTA, the Curie temperature for ferromagnetic ordering using
a modified Inden model.
\item NTA, the Neel temperature for antiferromagnetic ordering using a
modified Inden model.
\item IBM\&C, the individual Bohr magneton number for constituent C
using a modified Inden model.  For example IBM\&FE(BCC,FE) is the Bohr
magneton number for BCC Fe.  The identifier IBM\&FE(BCC,CR) means the
Bohr magneton number of a single Fe atom in BCC Cr.  An identifier
IBM\&FE(BCC,CR,FE) can be used to decribe the composition dependence of
the Bohr magneton number for Fe in BCC.
\item THET, the Debye or Einstein temperature.
\item MOBQ\&C, the logarithm of the mobility of constituent C
\item RHO, the electrical resistivity
\item MAGS, the magnetic suseptibility
\item GTT, the glas transition temperature
\item VISC, the viscosity
\item LPAX, the lattic parameter in X direction
\item LPAY, the lattic parameter in Y direction
\item LPAZ, the lattic parameter in Z direction
\item LPTH, the deviation from cubic structure
\item EC11A, the elastic constrant C11
\item EC12A, the elastic constrant C12
\item EC44A, the elastic constrant C44
\end{itemize}

\subsection{Constituent array and degrees}

A constituent array specifies one or more constituent in each
sublattice.  A constituent must be entered as a species with fixed
stoichiometry.  Between constituents in different sublattices one must
give a colon, ":", between interacting constituents in the same
sublattice one must give a comma, ",".  A constituent array with
exactly one constituent in each sublattice is also called an
``endmember'' as it give the value for a ``compound'' with fixed
stoichiometry.  Constituent arrays with one or more interaction
describe the composition dependence of the property, without such
parameter the property will vary liearly between the endmembers.

If there are no sublattices, like in the gas, one just give the phase
and the constituent

G(gas,C1O2)

If no degree is specified it is assumed to be zero.  For endmembers
the degree must be zero but it may sometimes be useful to specify the
zero in order to distinguish the parameter from the expression for the
chemical potential of a component.  In the gas phase one normally
assumes there are no interactions but it is possible to add such
parameters.  For an fcc phase with 4 sublattice for ordering and one
for interstitials an endmember parameter is

G(fcc,AL:NI:NI:NI:VA)

This would be the Gibbs energy of an Al1NI3 compound.

An interaction between vacancies and carbon in the austenite is

G(fcc,Fe:C,VA;0)

For interaction one should always specify a degree but also in this
case an omitted degree is interpreted as zero.

%
% "identifier" ( "phase name" , "constituent array" ; "degree" ) "expression"
%       ``reference''
%
%%%%%%%%%%%%%%%%%%%%%%%%%%%%%%%%%%%%%%%%%%%%%%%%%%%%%%%%%%%%%%%%%%%%%%%%%
%
% Below is an extract of the OC command interface.
% Comparing this with the current software is be way to know 
% where to update the user guide.
%
%F ! basic commands
%F   character (len=16), dimension(ncbas) :: cbas=&
%F        ['AMEND           ','CALCULATE       ','SET             ',&
%F         'ENTER           ','EXIT            ','LIST            ',&
%F         'QUIT            ','READ            ','SAVE            ',&
%F         'HELP            ','INFORMATION     ','BACK            ',&
%F         'NEW             ','MACRO           ','ABOUT           ',&
%F         'DEBUG           ','SELECT          ','DELETE          ',&
%F         'STEP            ','MAP             ','PLOT            ',&
%F         'HPCALC          ','FIN             ','                ']
%F ! in French
%F !        'MODIFIEZ        ','CALCULEZ        ','REGLEZ          ',&
%F !        'ENTREZ          ','EXIT            ','AFFICHER        ',&
%F !        'QUIT            ','LIRE            ','SAUVGARDE       ',&
%F !        'AIDEZ           ','INFORMATION     ','RETURNEZ        ',&
%F !        'NOUVEAU         ','MACRO           ','ABOUT           ',&
%F !        'DEBUG           ','SELECTIONEZ     ','EFFACEZ         ',&
%F !        'STEP            ','MAP             ','DESSINEZ        ',&
%F !        'HPCALC          ','FIN             ','                ']
%F ! options preceeded by -  
%F ! for example "list -out=myfile.dat all_data" or
%F ! "list all_data -out=myfile.dat"
%F   character (len=16), dimension(ncopt) :: copt=&
%F        ['OUTPUT          ','ALL             ','FORCE           ',&
%F         'VERBOSE         ','SILENT          ','                ']
%F !-------------------
%F ! subcommands to LIST
%F   character (len=16), dimension(nclist) :: clist=&
%F        ['DATA            ','SHORT           ','PHASE           ',&
%F         'STATE_VARIABLES ','BIBLIOGRAPHY    ','PARAMETER_IDENTI',&
%F         'AXIS            ','TPFUN_SYMBOLS   ','QUIT            ',&
%F         '                ','EQUILIBRIA      ','RESULTS         ',&
%F         'CONDITIONS      ','SYMBOLS         ','LINE_EQULIBRIA  '] <<<
%F!-------------------
%F! subsubcommands to LIST DATA
%F    character (len=16), dimension(nlform) :: llform=&
%F        ['SCREEN          ','TDB             ','MACRO           ',&
%F         'LATEX           ','                ','                ']
%F !-------------------
%F ! subsubcommands to LIST PHASE
%F   character (len=16), dimension(nclph) :: clph=&
%F        ['DATA            ','CONSTITUTION    ','MODEL           ',&
%F         '                ','                ','                ']
%F !-------------------
%F ! subcommands to CALCULATE
%F   character (len=16), dimension(ncalc) :: ccalc=&
%F        ['TPFUN_SYMBOLS   ','PHASE           ','NO_GLOBAL       ',&
%F         'TRANSITION      ','QUIT            ','GLOBAL_GRIDMIN  ',& <<
%F         'SYMBOL          ','EQUILIBRIUM     ','ALL_EQUILIBRIA  ']
%F !-------------------
%F ! subcommands to CALCULATE PHASE
%F   character (len=16), dimension(nccph) :: ccph=&
%F        ['ONLY_G          ','G_AND_DGDY      ','ALL_DERIVATIVES ']
%F !-------------------
%F ! subcommands to ENTER
%F   character (len=16), dimension(ncent) :: center=&
%F        ['TPFUN_SYMBOL    ','ELEMENT         ','SPECIES         ',&
%F         'PHASE           ','PARAMETER       ','BIBLIOGRAPY     ',&
%F         'CONSTITUTION    ','EXPERIMENT      ','QUIT            ',&
%F         'EQUILIBRIUM     ','SYMBOL          ','OPTIMIZE_COEFF  ',&
%F         'COPY_OF_EQUILIB ','                ','                ']
%F !-------------------
%F ! subcommands to READ
%F   character (len=16), dimension(ncread) :: cread=&
%F        ['UNFORMATTED     ','TDB             ','QUIT            ',&
%F         'DIRECT          ','                ','                ']
%F!-------------------
%F! subcommands to SAVE
%F    character (len=16), dimension(ncsave) :: csave=&
%F         ['UNFORMATTED     ','TDB             ','MACRO           ',&
%F         'DIRECT          ','LATEX           ','QUIT            ']
%F !-------------------
%F ! subcommands to AMEND first level (very few implemented)
%F ! many of these should be subcommands to PHASE
%F   character (len=16), dimension(ncam1) :: cam1=&
%F        ['SYMBOL          ','ELEMENT         ','SPECIES         ',&
%F         'PHASE           ','PARAMETER       ','BIBLIOGRAPHY    ',&
%F         'TPFUN_SYMBOL    ','CONSTITUTION    ','QUIT            ',&
%F         'COMPONENTS      ','GENERAL         ','DEBYE_MODEL     ']
%F !-------------------
%F ! subsubcommands to AMEND PHASE
%F   character (len=16), dimension(ncamph) :: camph=&
%F        ['MAGNETIC_CONTRIB','COMPOSITION_SET ','DISORDERED_FRACS',&
%F         'GLAS_TRANSITION ','QUIT            ','DEFAULT_CONSTIT ',&
%F         'DEBYE_MODEL     ','EINSTEIN_CP_MDL ','INDEN_WEI_MAGMOD',&
%F         'ELASTIC_MODEL_A ','                ','                ']
%F !-------------------
%F ! subcommands to SET.  
%F   character (len=16), dimension(ncset) :: cset=&
%F        ['CONDITION       ','STATUS          ','ADVANCED        ',&
%F         'LEVEL           ','INTERACTIVE     ','REFERENCE_STATE ',&
%F         'QUIT            ','ECHO            ','PHASE           ',&
%F         'UNITS           ','LOG_FILE        ','WEIGHT          ',&
%F         'NUMERIC_OPTIONS ','AXIS            ','INPUT_AMOUNTS   ',&
%F         'VERBOSE         ','AS_START_EQUILIB','                ']
%F ! subsubcommands to SET STATUS
%F   character (len=16), dimension(ncstat) :: cstatus=&
%F        ['ELEMENT         ','SPECIES         ','PHASE           ',&
%F         'CONSTITUENT     ','                ','                ']
%F !        123456789.123456---123456789.123456---123456789.123456
%F ! subsubcommands to SET ADVANCED
%F   character (len=16), dimension(ncadv) :: cadv=&
%F        ['EQUILIB_TRANSF  ','QUIT            ','                ']
%F !        123456789.123456---123456789.123456---123456789.123456
%F ! subsubcommands to SET PHASE
%F   character (len=16), dimension(nsetph) :: csetph=&
%F        ['QUIT            ','STATUS          ','DEFAULT_CONSTITU',&
%F         'AMOUNT          ','BITS            ','                ']
%F !        123456789.123456---123456789.123456---123456789.123456
%F !-------------------
%F ! subsubsubcommands to SET PHASE BITS
%F   character (len=16), dimension(nsetphbits) :: csetphbits=&
%F        ['FCC_PERMUTATIONS','BCC_PERMUTATIONS','IONIC_LIQUID_MDL',&
%F         'AQUEOUS_MODEL   ','QUASICHEMICAL   ','FCC_CVM_TETRADRN',&
%F         'FACT_QUASICHEMCL','NO_AUTO_COMP_SET','                ',&
%F         '                ','                ','                ',&
%F         '                ','                ','                ']
%F !        123456789.123456---123456789.123456---123456789.123456
%F!-------------------
%F! subcommands to STEP                                              <<<
%F    character (len=16), dimension(nstepop) :: cstepop=&
%F         ['NORMAL          ','SEPARATE        ','QUIT            ',&
%F          'CONDITIONAL     ','                ','                ']
%F!         123456789.123456---123456789.123456---123456789.123456
%F !-------------------
%F ! subcommands to DEBUG
%F   character (len=16), dimension(ncdebug) :: cdebug=&
%F        ['FREE_LISTS      ','STOP_ON_ERROR   ','ELASTICITY      ',&
%F         '                ','                ','                ']
%F !-------------------
%F ! subcommands to SELECT, maybe some should be CUSTOMMIZE ??
%F   character (len=16), dimension(nselect) :: cselect=&
%F        ['EQUILIBRIUM     ','MINIMIZER       ','GRAPHICS        ',&
%F         'LANGUAGE        ','                ','                ']
%F !-------------------
%F ! subcommands to DELETE
%F   character (len=16), dimension(nrej) :: crej=&
%F        ['ELEMENTS        ','SPECIES         ','PHASE           ',&
%F         'QUIT            ','COMPOSITION_SET ','EQUILIBRIUM     ']
%F !-------------------
%F ! subcommands to PLOT OPTIONS
%F   character (len=16), dimension(nplt) :: cplot=&
%F         ['PLOT            ','XRANGE          ','YRANGE          ',&
%F         'XTEXT           ','YTEXT           ','TITLE           ']
%F !-------------------
%F
%
%! end extract of command user interface
%%%%%%%%%%%%%%%%%%%%%%%%%%%%%%%%%%%%%%%%%%%%%%%%%%%%%%%%%%%%%%%%%%%%%%%%%
%
% below the commands are arranged in alphabetical order

\section{All commands}

The commands in alphabetica order as listed with the ?

\begin{tabular}{llll}
ABOUT           & ENTER           & LIST           & READ    \\
AMEND           & EXIT            & MACRO          & SAVE    \\
BACK            & FIN             & MAP            & SELECT  \\
CALCULATE       & HELP            & NEW            & SET     \\
DEBUG           & HPCALC          & PLOT           & STEP    \\
DELETE          & INFORMATION     & QUIT                     \\
\end{tabular}

Many of the commands have ``subcommands'' and usually there is a
default (listed within slashes //) which is selected by pressing
return.  One can type commands and subcommands and other parameters on
the same line if one knows the order, using a comma, ``,'' to select
the default.

There some options that can be set for the whole session or for just a
single command.  The options are idenfified by a - in front like
-output=myfile.dat.

\subsection{Option}

These should be possible to specify at each command.  But
they are not yet implemented.

\begin{itemize}
\item -OUTPUT {\em file name}
\item -ALL apply for all
\item -FORCE override normal restrictions
\item -VERBOSE write information while executing
\item -SILENT do not write anything except fatal error messages
\end{itemize}
%===================================================================
\section{About}

Some information about the software.

%===================================================================
\section{Amend}

Intended to allow changes of already entered data. Only some
of the subcommands are implemented.

%--------------------------------
\subsection{Element}

Not implemented yet.

%--------------------------------
\subsection{Debye\_Model}

Not implemented yet.

%--------------------------------
\subsection{Components}

By default the elements are the components.  Ths command can set any
orthogonal set of species as components.  The number of components
cannot be changed.

Not implemented yet.

%--------------------------------
\subsection{Constitution}

The user can set a constitution of a phase before a calculation.  This
will be used as initial constitution for a calculation.

%--------------------------------
\subsection{General}

A number of user specific settings for defaults can be made:

\begin{itemize}
\item The name of the system.

\item The level of the user (beginner, frequent user, expert).  This
may affect the behaviour of the program.

\item If global minimization is allowed or not.

\item If gridpoints should be merged after global minimization.  By
default not.

\item  Automatic creation or deletion of composition sets not allowed.
\end{itemize}

%--------------------------------
\subsection{Parameter}

The possible parameters are defined by the model of the phase.  By
specifying a parameter the user can change its expression.  See the
ENTER PARAMETER command.  Not implemented yet.

%--------------------------------
\subsection{Phase}

Some of the properties of the phase can be amended by this command.

%.........................
\subsubsection{Magnetic\_Contrib}

A model for the magnetic contribution to the Gibbs energy can be set
by this command.

%.........................
\subsubsection{Composition\_Set}

More composition sets of a phase can be created or deleted.  Phases
with miscibility gaps or which can exist with different chemical
ordering like A2 and B2 must be treated as different composition sets.
The user can specify a prefix and suffix for the composition set.  The
composition set will always habe a suffix \#digit where digiit is a
number between 1 and 9.  One cannot have more than 9 composition sets.

Composition sets can also be created automatically by the software.  In
such a case the composition set will have the suffix \_AUTO,

In some cases it may be interesting to calculate metastable states inside
miscibility gaps and one can prevent automatic creation of composition
sets by {\rm AMEND GENERAL} or for an individual phase by {\em SET
PHASE BIT {\em phase} NO\_AUTO\_COMP\_SET}

%.........................
\subsubsection{Default\_Constit}

The default constitution of a phase can be set.  This will be used
when the first calculation with the phase is made and sometimes if
there are convergence probems.  Depending on the minimizing software
used the initial consititution can be important to find the correct
quilibrium if the phase has ordering or a miscibility gap.

%.........................
\subsubsection{Disordered\_Fracs}

For phases with several sublattices the Gibbs energy of the phase can
be divided into two sets of fractions where the second or
``disordered'' set have only one or two sublattice and the fractions
on these represent the sum of fraction on some or all of the first or
``ordered'' set of sublattices.  This is particularly important for
phases with ordering like FCC, BCC and HCP and for intermediate phases
like SIGMA, MU etc.

%.........................
\subsubsection{Glas\_Transition}

Not implemented yet.

%.........................
\subsubsection{Quit}

Do not amend anything for the phase.

%--------------------------------
\subsection{Quit}

Do not amend anything.

%--------------------------------
\subsection{Bibliography}

The bibliographic reference for a parameter can be amended.

%--------------------------------
\subsection{Species}

Not implemented yet.

%--------------------------------
\subsection{Symbol}

Not implemented yet.

%--------------------------------
\subsection{Tpfun\_Symbol}

You can replace a TP function with a new expression.

This is somewhat dangerous if you have several equilibria because each
equilibria has its own list of most recently calculated values of the
function and they may not be aware of a change of the function and go
on using the already calculated value unless you change $T$ or $P$, in
eqch equilibrium, which will force recalculation.  I am thinking of a
way to handle this.

%===================================================================
\section{Back }

Return back from the command monitor to the application program.  In
the OC software itself it means terminate the program.

%===================================================================
\section{Calculate }

Different things can be calculated.  The normal thing to calculate is
{\bf equilibrium}, the other things are special.

%--------------------------------
\subsection{All\_Equilibria}

Intended for the assessment procedure.  Not implemented yet.

%--------------------------------
\subsection{Equilibrium}

The normal command to calculate the equilibrium of a system for the
current set of conditions and phase status.  You can calculate a
metastable equilibrium if some phases that should be stable have been
set dormant or suspended or if automatic creation of composition sets
is not allowed.

%--------------------------------
\subsection{Global\_Gridmin}

Calculate with the global grid minimizer without using this result as a
start point for the general minimizer.  Used to debug the grid
minimizer.

%--------------------------------
\subsection{No\_Global}

Calculate the equilibrium with the current minimizer without using a
global gid minimizer to generate start constitutions.  The current
equilibrium is used as start point.  Can be quicker when just a small
change of conditions made since previous calculation.  It means no
check of new miscibility gaps.

%--------------------------------
\subsection{Phase}

The Gibbs energy of a phase and possible derivatives are calculated.
Mainly for debugging the implementation of models.

\subsubsection{Only\_G}

The Gibbs energy and all T and P derivatives calculated and listed.

\subsubsection{G\_and\_dGdy}

The Gibbs energy, all T and P derivatives and all first
derivatives with respect to constituents are calculated and listed.

\subsubsection{All\_Derivatives}

The Gibbs energy, all T and P derivatives and all first and second
derivatives with respect to constituents are calculated and listed.

%--------------------------------
\subsection{Quit}

Quit calculating.

%--------------------------------
\subsection{Symbol}

A state variable symbol or function is calculated using the results
from the last equilibrium or grid minimizer calculation.  It is used
in particular for calculation of ``dot derivatives'' like H.T for the
heat capacity.

%--------------------------------
\subsection{Tpfun\_Symbols}

All or a specific TPFUN symbol is calculated for current values of T
and P.

%===================================================================
\section{Debug }

Several possibilities to trace calculations will be implemented in
order to find errors.  The only implemented feature is to stop the
program whenever an error occurs.  This is useful to find errors using
macro files so the macro not just goes on doing other things.

\subsection{Stop\_on\_Error}

The program will stop at the command level after printing the error
message if an error has occured when using macro file.  This makes it
easier to use macro files to find errors.

%===================================================================
\section{Delete }

Not implemented yet and may never be, it is not so easy to allow
deleting things when the data structure is so involved, it may be
better to enter the data again without the data that should be
deleted.

\section{Exit }

Terminate the OC software.

%===================================================================
\section{Enter }

In most cases data will be read from a database file.  But it is
possible to enter all thermodynamic data interactivly.  This should
normally start by entering all elements, then all species (the
elements will automatically also be species) and then the phases.

A species have a fixed stoichiometry and possibly a charge.  The
species are the constituents of the phases.

A phase can have sublattices and various additions like magnetic or
elastic (the latter not implemented yet).  

TPFUN symbols can be used to describe common parts of model
parameters.

Each parameter of a phase is entered separately.  One may use
TPFUN symbols which are already entered.

At present the multicomponent CEF model and the ionic 2-sublattice
liquid model are the only one implemented.  This includes the gas
phase, regular solutions with Redlich-Kister Muggianu model and phases
with up to 10 sublattices and magnetic contributions.

%--------------------------------
\subsection{Constitution}

The constitution (fraction of all constituents) of a phase can be
entered.  This is a way to provide start values for a calculation or
to calculate the Gibbs enegy for a specific phase at a specific
constitution using {\bf calculate phase}.

%--------------------------------
\subsection{Element}

The data for an element is entered.  It consists of is symbol, name,
reference state, mass, H298-H0 and S298.  The latter two values are 
never used for any calculation.

%--------------------------------
\subsection{Equilibrium}

One can have several equilibria each with a unique set of conditions
incuding phase status (dormant, suspended, fix or entered).  This is
useful for compare different states, to simulate transformations and
to assess model parameters as each experimental or theoretical
information represented as an equilibrium.  

Each equilibrium is independent and they can be calculated in
parallel.

%--------------------------------
\subsection{Experiment}

This is for assessment, not implemented yet.

%--------------------------------
\subsection{Parameter}

A parameter is definded by its identifier, the phase and constituent
array.  A parameter can be a constant or depend on T and P.  The
parameter will be multiplied with the fractions of the constituents
given by its constituent array.

For example G(LIQUID,CR) is the Gibbs energy of liquid Cr relative to
its reference state, normally the stable state of Cr at 298.15 K and 1
bar.

For a gas molecule G(GAS,C1O2) is the Gibbs energy of the C1O2 molecule
relative to the reference states of C (carbon) and O (oxygen).

For phases with sublattices the constituents in each sublattice are
separated by a semicolon, ``:'' and interacting constituents in
the same sublattice by a comma, ``,''.  For example

G(FCC,FE:C,VA) is the interaction between C (carbon) and VA (vacant
interstitial sites) in the FCC phase.

One can store many different types of data in OC using the parameter
identifier.  A description of the identifiers currently implemeneted
are given in the introduction.  Here is a short list.

\begin{itemize}
\item G, the Gibbs energy or an interaction parameter
\item TC, the critical temperature for ferro or antiferro magnetic ordering
\item BMAGN, the avarage Bohr magneton number
\item CTA, the Curie temperature for ferromagnetic ordering
\item NTA, the Neel temperature for antiferromagnetic ordering
\item IBM\&C, the individual Bohr magneton number for constituent C
\item THETA, the Debye or Einstein temperature
\item MOBQ\&C, the logarithm of the mobility of constituent C
\item RHO, the electrical resistivity
\item MAGS, the magnetic suseptibility
\item GTT, the glas transition temperature
\item VISC, the viscosity
\item LPAX, the lattice parameter in X direction
\item LPTH, the deviation from cubic structure
\item EC11A, the elastic constrant C11
\item EC12A, the elastic constrant C12
\item EC44A, the elastic constrant C44
\end{itemize}

The current list can be obtained by the command LIST PARAMETER\_ID.

%--------------------------------
\subsection{Phase}

All thermodynamic data are connected to a phase as defined by its
parameters, see {\bf enter parameter}.  A phase has a name with can
contain letters, digits and the underscore character.

A phase can have 1 or more sublattices and the user must specify the
number of sites on each.  He must also specify the constituents on
each sublattice.  For some models, like the ionic liquid model, the
number of sites may change with composition.

By default the model for a phase is assumed to be the Compound Energy
Formalism (CEF).  If any onther model should be used that is set by
the {\bf amend} or {\bf set phase bit} commands.

%--------------------------------
\subsection{Quit}

Quit entering things.

%--------------------------------
\subsection{Bibligraphy}

Each parameter must have a reference.  When entering a parameter a
reference symbol is given and with this command one can give a full
reference text for that symbol like a published paper or report.

%--------------------------------
\subsection{Species}

A species consists of a name and a stochiometric formula.  It can have
a valence or charge.  The name is often the stoichiometric formula
but it does not have to be that.  Examples:

\begin{itemize}
\item enter species water h2o
\item enter species c2h2cl1\_trans c2h2cl2
\item enter species c2h2cl1\_cis c2h2cl2
\item enter species h+ h1/- -1
\end{itemize}

Single letter element names must be followed by a stocichiometric
factor unless it is the last element when 1 is assumed.  Two-letter
element names have by default the stoichiometric factor 1.

\begin{itemize}
\item enter species carbonmonoxide c1o1
\item enter species cobaltoxide coo
\item enter species carbondioxide c1o2
\end{itemize}

The species name is important as it is the name, not the
stoichiometry, that is used when referring to the species elsewhere
like as constituent.

%--------------------------------
\subsection{Symbol}

The OC package has both ``symbols'' and ``tpfun\_symbols'', the latter
has a very special syntax and can be used when entering parameters.

The symbols are designed to handle relations between state variables,
one can define expressions like 

enter symbol K = X(LIQUID,CR)/X(BCC,CR);

where K is set to the partition of the Cr mole fractions between
liquid and bcc.

The symbols also include ``dot derivatives'' like H.T which is the
temperature derivative of the enthalpy for the current system at the
given set of conditions, i.e. the heat capacity.

%--------------------------------
\subsection{Tpfun\_Symbol}

This symbol is an expression depending on T and P that can be used
when entering parameters.  A TPfun can refer to another TPfun.

TPFUNS have a strict syntax because the software must be able to
calculate first and second derivatives with respect to $T$ and $P$.

%===================================================================
\section{Exit}

Terminate the OC software in English.

%===================================================================
\section{Fin }

Terminate the OC software in French.

%===================================================================
\section{Help }

Can give a list if commands or subcommands or parts of this help text.

%===================================================================
\section{HPCALC }

A reverse polish calculator.

%===================================================================
\section{Information }

Not implemeneted yet.

%===================================================================
\section{List }

Many things can be listed.  Output is normally on the screen unless it
is redirected by the -output option.

%--------------------------------
\subsection{Axis}

Lists the axis set by the user.

%--------------------------------
\subsection{Conditions}

Lists the conditions set by the user.

%--------------------------------
\subsection{Data}

Lists all thermodynamic data.

%--------------------------------
\subsection{Equilibria}

Lists the equilibria entered (not the result ...).

%--------------------------------
\subsection{Phase}

List data for a phase

%...................
\subsubsection{Data}

List the model and thermodynamic data.

%...................
\subsubsection{Constitution}

List the constitution of the phase.

%...................
\subsubsection{Model}

List some model data for example if there is a disordered fraction set.

%--------------------------------
\subsection{Quit}

You did not really want to list anyting.

%--------------------------------
\subsection{Bibliography}

List the bibliographic references for the data.

%--------------------------------
\subsection{Results}

List the results of an equilibrium calculation.  This is the most
frequent list command.  The listing will contain the current set of
conditions, a table with global data, a table with component specific
data and then a list of stable phases with amounts, compositions and
possibly constitutions.  It is possible to list also unstable phases.

There are 9 options for the formatting:
\begin{enumerate}
\item Output in mole fractions, phase constituents in value order
  (constituent with highest fraction first)
\item As 1 but include also the phase constitution (sublattices and
  their fractions)
\item As 1 with the phase composition in alphabetical order (do not work)
\item As 1 with mass fractions
\item As 4 with the phase composition in alphabetical order (do not work)
\item As 4 and also include the phase constitutions
\item All phases will be listed with composition in mass fractions and
  in alphabetical order of the elements.  A negative driving force for
  a phase means the phase is not stable.
\item All phases will be listed with composition in mole fraction in
  value order and driving force, negative driving force means the
  phase is not stable.
\item All phases will be listed with composition and constitutions in
  alphabetical order of the elements and the driving force.
\end{enumerate}

For each phase the name, status and driving force (in dimensionless
units) is given on the first line.  

The second line has the amount of the phase in moles and mass of
components (zero if not stable) and its volume (zero if no pressure
data).  

The third line has the number of formula units of the phase (zero if
not stable) and the moles of atoms per formula unit.  The first value
on the third line multiplied with the second will be the first value
on the second line.  The gas phase and phases with interstitials have
a varying amount of moles of atoms per formula units.

%--------------------------------
\subsection{Short}

A listing with a single line for each element, species and phase with
some essential data.

%--------------------------------
\subsection{State\_Variables}

Values of state variables like G, HM(LIQUID) etc. can be listed.
Terminated by an empty line.  Note that symbols cannot be listed here,
they are calculated by the CALCULATE SYMBOL command.

%--------------------------------
\subsection{Symbols}

All state variable symbols listed but not their values, they are
calculated by the CALCULATE SYMBOL command.

%--------------------------------
\subsection{Tpfun\_Symbols}

All TPFUN symbols listed.

%===================================================================
\section{Macro }

By specifying a file name commands will be read from that file.  The
default extention is OCM.  A macro file can open another macro file
(max 5 levels).  When a mcro file finish with SET INTERACTIVE the
calling macro file will continue.

%===================================================================
\section{Map }

For phase diagram calculations.  One must first ste two axis with
state variables also set as conditions.

If one gives several MAP commands one can erase or keep the previous
results.

During mapping each calculated equilibria is saved and then different
kinds of state vaiables can be used for plotting.

%===================================================================
\section{New }

To remove all data so a new system can be entered.  Fragile

%===================================================================
\section{Plot }

Plot the result from a STEP or MAP calculation.  A simple interface to
gnuplot has been implemented.  One can select state variables for the
plot axis and some options like range and title.

%===================================================================
\section{Quit }

Terminate the OC software in Swedish.

%===================================================================
\section{Read }

At present there a very limited SAVE command implemented in OC as it
is difficult to do that before the datastructure is well defined.

It is possible to read a (non-encrypted) TDB file but it should be not
too different from what is generated by the LIST\_DATA command in TC.

%--------------------------------
\subsection{Quit}

You did not really want to read anything.

%--------------------------------
\subsection{TDB}

A TDB file (with extention TDB) should be specified.  The TDB file
must not deviate very much from the output of Thermo-Calc.

%--------------------------------
\subsection{Unformatted}

For use to read a file created with a SAVE UNFORMATTED command.  It
will not always work as the datastructure is not fixed.

%===================================================================
\section{Save }

There are several forms of save, three forms write a text file that
can be read and modified with a normal editor.  Two forms are
unformatted, either on a sequential file or a direct (random access)
file.

%--------------------------------
\subsection{Direct}

It will eventually be possible to save the result of STEP and MAP
commands on a random access file for later processing.

%--------------------------------
\subsection{LaTeX}

The thermodynatic data will be formatted according to LaTeX for later
inclusion in publications.  Not implemented.

%--------------------------------
\subsection{Macro}

The thermodynamic data will be written as a macro file that can later
be read back into the OC software.  Not implemented.

%--------------------------------
\subsection{Quit}

You did not want to save.

%--------------------------------
\subsection{TDB}

The thermodynamic data will be written in a text form that can be later
read by OC or other software.  Not implemented.

%--------------------------------
\subsection{Unformatted}

The intention is that one will be able to save the current status of
the calculations on a file and then reassume the calculations by
reading this file.  A tentative version is implemented.


%===================================================================
\section{Select }

%--------------------------------
\subsection{Equilibrium}

As the user can enter several equilibria with different conditions
this command allows him to select the current eqilibria.

%--------------------------------
\subsection{Graphics}

Not implemented yet.

%--------------------------------
\subsection{Minimizer}

Not implemented yet

%===================================================================
\section{Set }

Many things can be set.  Things to be ``set'' and ``amended''
sometimes overlap.

%--------------------------------
\subsection{Advanced}

Not implemented yet

%--------------------------------
\subsection{Axis}

A condition can be set as an axis variable with a low and high limit
and a maximum increement.  With 2 or more axis one will calculate a
phase diagram, i.e. lines where the set of stable phases changes.

With one axis one calculates the set of stable phases and their
properties while changing the axis variable.

%--------------------------------
\subsection{Condition}

A condition is a value assigned to a state variable or an expression
of state variables.  By setting the status of a phase to fix one has
also set a condition.

%--------------------------------
\subsection{Echo}

This is useful command in macro files.

%--------------------------------
\subsection{Input\_Amounts}

This allows the user to specify a system by giving a redundant amount
of various species in the system.  The software will tranform this to
conditions on the amounts of the components.

%--------------------------------
\subsection{Interactive}

The usual end of a macro file.  Gives command back to the keyboard of
the user, or to the calling macro file.  Without this the program will
just terminate.

%--------------------------------
\subsection{Level}

I am no longer sure what this should do and if it is needed ...

%--------------------------------
\subsection{Log\_File}

A useful command to save all interactive input while running OC.  The
log file can easily be transformed to a macro file.  All bug reports
should be accompanied by a log file which reproduces the bug.

%--------------------------------
\subsection{Numeric\_Options}

Some numeric option can be set.

%--------------------------------
\subsection{Phase}

Some phase specific things can be set, also for the model.

%....................
\subsubsection{AMOUNT}

One can specify the amount of the phase which is used as initial value
for an equilibrium calculation.

%....................
\subsubsection{BITS}

Some of the models and data storage depend on the bits of the phase.
These are

%. . . . . . . . . .
\begin{itemize}
%\subsubsubsection{FCC\_PERMUTATIONS]
\item FCC\_PERMUTATIONS is intended for the 4 sublattice CEF model for
fcc ordering.  Setting this bit means that only unique model
parameters needs to be entered, the software will take care of all
permutations.  HCP permutations is also handelled by this bit as they
are identical in the 4 sublattice model.

%. . . . . . . . . .
%\subsubsubsection{BCC\_PERMUTATIONS}
\item BCC\_PERMUTATIONS is intended for the 4 sublattice CEF model for
BCC ordering.  The BCC tetrahedron is unsymmetric which makes it a bit
more complicated.  Not implemented yet.

%. . . . . . . . . .
%\subsubsubsection{IONIC\_LIQUID\_MDL}
\item IONIC\_LIQUID\_MDL.  By setting this bit the phase is treated
with the 2 sublattice paritally ionic liquid model.  It must have been
entered with 2 sublattices and only cations in the first sublattice
and only anions, vacancy and neutrals in the second.

%. . . . . . . . . .
%\subsubsubsection{AQUEOUS\_MODEL}
\item AQUEOUS\_MODEL. Not implemented yet.

%. . . . . . . . . .
%\subsubsubsection{QUASICHEMICAL}
\item QUASICHEMICAL. Is intended for the classical quasichemical
model for crystalline phases.  Not implemented yet.

%. . . . . . . . . .
%\subsubsubsection{FCC\_CVM\_TETRADRN}
\item FCC\_CVM\_TETRADRN.  Is intended for the CVM tetrahedron model.
Not implemented yet.

%. . . . . . . . . .
%\subsubsubsection{FACT\_QUASICHEMCL}
\item FACT\_QUASICHEMCL.  Is intended for one for the FACT modified
quasichemical liquid models.  Not implemented yet.

%. . . . . . . . . .
%\subsubsubsection{NO\_AUTO\_COMP\_SET}
\item NO\_AUTO\_COMP\_SET.  This makes it possible to prevent that a
specific phase has automatic composition set created during
calculations.

%. . . . . . . . . .
%\subsubsubsection{ELASTIC\_MODEL\_A}
\item ELASTIC\_MODEL\_A.  This should specify the elastic model to be
used.  Not implemented yet.
\end{itemize}

%....................
\subsubsection{CONSTITUTION}

This is the same as {\bf amend phase constitution}.

%....................
\subsubsection{DEFAULT\_CONSTITU}

Same as {\bf amend phase default\_constit}.

%....................
\subsubsection{STATUS}

A phase can have 4 status

\begin{itemize}
\item entered, this is the default.  The phase will be stable if that
would give the most stable state for the current conditions.  The user
can give a tentative amount.
\item suspended, the phase will not be included in any calculations.
\item dormant, the phase will be included in the calculations but will
not be allowed to become stable even if that would give the most
stable equilibrium.  In such a case the phase will have a positive
driving force.
\item fixed means that it is a condition that the phase is stable with
the specified amount.  Note that for solution phases the composition
is not known.
\end{itemize}

%--------------------------------
\subsection{Quit}

You did not really want to set anything

%--------------------------------
\subsection{Reference\_State}

For each component (also when not the elements) one should be able to
specify a phase at a given temperature and pressure as reference
state.  The phase must exist for the pure component.

%--------------------------------
\subsection{Status}

%....................
\subsection{Constituent}

A constituent of a phase can be suspended.  Not yet implemented.

%....................
\subsection{Element}

An element can be ENTERED or SUSPENDED.  If an element is suspended
all species with this element is automatically suspended.

%....................
\subsection{Phase}

A phase can have 4 status as described for the SET PHASE STATUS
command above.  Changing the pase status does not affect anything
except the phase itself.

%....................
\subsection{Species}

A species can be ENTERED or SUSPENDED.  If a species is suspended
all phases that have this as single constituent in a sublattice
will be automatically suspened.

%--------------------------------
\subsection{Units}

For each property the unit can be specified like Kelvin, Farenheit or
Celsius for temperature.  Not implemented yet.

%--------------------------------
\subsection{Weight}

Intended for assessments.  Not implemented yet.

%===================================================================
\section{Step }

Requires that a single axis is set.

Calculates equilibria from the low axis limit to the high at each
increment.  

%===================================================================
% Using this file for on-line help there must be a section after last command
\section{Summary }

That's all.

\end{document}
