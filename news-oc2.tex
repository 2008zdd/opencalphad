\documentclass[12pt]{article}
\usepackage[latin1]{inputenc}
\usepackage{graphicx,subfigure}
\topmargin -1mm
\oddsidemargin -1mm
\evensidemargin -1mm
\textwidth 165mm
\textheight 220mm
\parskip 2mm
\parindent 3mm
%\pagestyle{empty}

\begin{document}

{\bf \Large New features in the Open Calphad software version 2}

\bigskip

Bo Sundman, \today

\section{Background}

The Open Calphad (OC) initiative started in 2010 when a group of
scientists decided that there was a need of a high quality open source
software to gain acceptance of computational thermodynamics (CT) as a
useful tool in materials science.  The use of thermodynamic
calculations in many applications is severely restricted by the cost
as well as the hardware and software requirements imposed by the
proprietary thermodynamic software.  Providing a free software would
simplify such implementations and open a much larger market also for
the high quality databases provided by the commercial vendors.

Another aim was to support the scientific interest in new
thermodynamic models and improved algorithms for multicomponent
thermodynamic calculations and a better software for thermodynamic
assessments as decrsibed in the book by Lukas et al.\cite{07Luk}.  At
present such developments can only be done by scientists who are
affiliated to the commercial software companies.

The current OC software is available on~\cite{ocweb}.  For software
collaborations there is also a repository called opencalphad
at~\cite{github}.

The OC software in its present state is mainly of interest for
researchers, scientists and students with programming skills.  In a
few years it may be as stable as the commercial software and can be
used also for teaching computational thermodynamics.

\section{Structure of the OC software}

The software is divided into packages.  There are well defined
software interfaces between the packages that makes it possible to
extend and change them independently.

\begin{itemize}
\item The General Thermodynamic Package. (GTP) which has data
  structures for storing model parameters, conditions and calculated
  results and code to calculate the Gibbs energy and its first and
  second derivatives of phase when the $T, P$ and constitution of the
  phase is known.

  As this was the first package developed it includes a number of
  general untility facilities needed also by the other packages:

\begin{itemize}
\item The TP function package for storing and calculating functions
  that depend on $T$ and $P$, including first and second derivatives.
\item The METLIB utility package mainly for use by the interactive
  user interface. Originally written in Fortran 77 and modified to the
  new Fortran standard but it includes features that are depreciated
  like ENTRY.
\item The command line user interface with a VAX/VMS flavour is part
  of the METLIB package.
\item The numlib routines for inverting a matrix and solving a system
  of linear equations.  Currently very old and stable but rather
  inefficient routines are used.
\end{itemize}

\item The HMS minimizer implementing the algorithm by
  Hillert\cite{81Hil} for finding the equilibrium state in a
  multicomponent system for many different kinds of external
  conditions.  It makes use of GTP for calculating the Gibbs energy
  and derivatives for each phase.

\item The step/map/plot (SMP) package for calculating and plotting
  diagrams.  It uses HMS for calculating equilibria for conditions
  varying along the axis and the free software GNUPLOT for plotting on
  various devices.

\item The OC-TQ software interface to integrate OC in general
  application software for various simulations.

\end{itemize}

\section{Features in version 1}

The version 1 release of OC in 2013 could calculate multicomponent
equilibra using Hillert's algorithm\cite{81Hil} for models based on
the Compound Energy Formalism (CEF)\cite{01Hil,07Luk}. It included a
possibility to read unencrypted TDB files and a simple command
interface with macro facilities to set conditions, calculate
equilibria and list results.  It has a grid minimizer to ensure
finding the global minimum and detect miscibility gaps.  There was
also a limited application software interface called OC-TQ.

\section{New features in version 2}

The most important new facilities since version 1 are generating
property and phase diagrams.  However, these and many of the other
features are still incomplete and fragile and may not work properly in
many cases.  Feedback from users (providing the data and a macro file
reproducing the problem) is the best way to obtain a more stable and
error free software.

A compiler for Fortran 95 like GNU gfortran 4.8 or later is required.

A new documentation of the code, a user guide and additional examples
as macro files is also be provded.  However, as a complete revision of
both data structures and subroutines are planned for version 3 the
docummentation is not fully up to date.

\begin{enumerate}
\item The STEP procedure for property diagram.  Such diagrams are
  calculated with a single axis variable and the user can calculate
  and plot how various state variables or model properties depend on
  the axis variable.  A primitive version of the step procedure was
  available also in version 1 but in the new version the exact value
  of the change of the set of stable phases is calculated.

  There is also a ``step separate'' option for Gibbs energy curves and
  similar things when each phase is calculated separately along the
  axis.

  There is a problem with the STEP procedure in a binary system using
  a composition as axis.  The STEP will stop at a phase boundary and
  it does not take into accound that nothing changes in a two-phase
  region except the amount of the phases.

\item The MAP procedure for phase diagrams.  This calculates lines
  where the set of stable phases changes for different values of the
  axis variables.  At present only two axis are allowed but in a
  future release up to 5 axis will be implemented.

  As the mapping has some problems to calculate all lines it is
  possible to execute several map commands and append to the previous
  results.  As an emergency one can remove lines that are wrong by
  editing the input file to GNUPLOT.

  Mapping of binary systems is fairly stable although there are
  problems at the top of miscibility gaps and crossing congruent
  transformations.  This can usually be handelled by several start
  points.

  Mapping of multicomponent system is possible but in general many
  lines are missing.  There is an unresilved problem to exit from
  certain node points.  Invariant equilibria in ternary or higher
  order systems is not implemented.

  The present version of mapping will not discover miscibility gaps.
  The phase diagram for Cr-Fe looks horrible.  Things like that will
  be taken care of in a future release.

  During both MAP and STEP all calculated equilibria are saved and it
  is possible to plot various properties.  All node points are saved
  as equilibria which can be inspected individually and it is also
  possible to copy equilibria along a line to a current equilibria and
  extract values.

  It is not possible to save the results from a STEP or MAP command on
  a file.  The user should create MACRO files for calculations he
  would like to repeat.

\item GNUPLOT version 4.6 or later is used for generate the graphics.
  In the user interface of OC some additional graphics options, like a
  title and ranges of the x and y axis, has been added.  It is also
  possible to edit the output files from OC to take advantage of all
  the graphics facilities of GNUPLOT.

%  GNUPLOT does not support triangular diagrams.

\item The ``dot derivative'' method to calculate derivatives of state
  variables has been implemented.  This allows calculation of
  properties like the heat capacity without resorting to numerical
  derivatives.  It makes use of the analytical first and second
  derivatives of the Gibbs energy for $T, P$ and all constituent
  fractions implemented in the model package.

  The implementation is not finished but derivatives of several state
  variables with respect to $T$ are available.

\item The ionic two-sublattice liquid model (I2SL) which can handle
  liquids with and without short range ordering is now implemented.

\item For the OC-TQ interface a new method has been introduced to
  identify phases called ``phase tuples''.  A phase tuple is a Fortan
  95 structure (TYPE) with two integer values, one for the phase and
  the other for the composition set.  The user interface of OC also
  use phase tuples when listing phases and composition sets.

  When a phase is entered it has one composition set with number 1 and
  a phase tuple is created with the same index as the phase and the
  composition set index equal to 1.  When a new composition set is
  entered for a phase, either by the user or by the software itself,
  for example the grid minimizer, the phase tuple index for the new
  composition set will be higher than any of the phases and have as
  values the phase number and a composition set number 2 or higher.

  There is an example calling the OC-TQ interface from C/C++.  It is
  rather clumsy and a better way to transfer data should be developed.

\item Minor things
\begin{itemize}
\item The user can select the reference state of the elements for the
  thermodynamic properties.  This should be done before any MAP or
  STEP command.

\item The partitioning of the Gibbs energy for phases with
  order/disorder transformations has been revised and simplified.

\item The grid minimizer for global equilibria has been improved
  handling phases with ions.

\item An UNFORMATTED output to a file of the data and results from a
  single equilibrium calculation is now possible and the file can be
  read back with all data and results.  It is not yet possible to save
  the result from a STEP or MAP calculation on a file.

\item The user can enter several equilibria for the same system and
  have different conditions in each and calculate them separately and
  transfer data between them.  This facility is used for storing
  step/map results and is a preparation for the software to assess
  model parameters from experimental data.  Each equilibrium is
  independent and they be calculated in parallel.

\item OC has a flexible way to handle properties like mobilities,
  elastic constants etc that may depend on the phase, $T, P$ and the
  phase constitution.  Some properties are predefined but a skilled
  programmer can easily add a specific property and a model to use it
  in a calculation.  The values of such properties can be obtained
  interactivly or by application software in the same way as
  thermodynamic state variables.

\item Reading TDB files is now less strict and it is possible to
  specify the elements to be selected from the database.  With some
  editing of the TDB file it is also possible to read data for ordered
  phases modelled with the Thermo-Calc partitioned method.

  There is still no output of TDB files from OC.  If you enter
  parameters interactivly and want to keep a copy of these on a text
  file use the ``set log'' command in advance.

\item Parallelization has been tested for the grid minimizer and for
  the calculation of the inverse phase matrices.  It has been
  indicated in the code where it can be useful to speed up other parts
  of the calculations.  A simple test of the parallelization of
  calulating and inverting of the phase matrices at each iteration
  reduced the time for some calculations by 25\%.

  Mapping of phase diagrams can also be done in parallel for all the
  lines.  For assessments all equilibria representing experimental
  data can be calculated in parallel.

\item Composition sets created automatically by the grid minimizer
  should normally be removed if they are not needed after the full
  equilibrium has been calculated.  

  If the user has created composition sets for phases that may appear
  with specific constitutions, like the cubic MC carbide as a
  composition set of the FCC phase, the OC software will try to assign
  the correct composition to the approriate composition set.

\item Phases with order/disorder transformations like FCC (L1$_2$ and
  L1$_0$) modelled with 4 sublattices can have an ``FCC\_PERMUTATION''
  bit set to simplify entering the parameters.  With this bit set the
  user needs to enter each unique parameter only once, not all the
  permutations.  All kinds of interaction parameters can be entered up
  to second order.  For the BCC ordering permutations are more
  complicated and has not yet been implemented.

\item Phases with LRO ordering (including phases with LRO that never
  disorder like $\sigma, \mu$ and Laves) one can have a disordered
  fraction set for parameters that depend on the overall composition
  of the phase but are independent of the phase constitution.
\end{itemize}

\end{enumerate}

\section{Known bugs and problems and features not yet implemented}

Some things are problemnatic and from the long list of things we wanted
to implement but did not manage this time, these are a few:

\begin{itemize}
\item Ternary isothermal sections are difficult to calculate and
  cannot be plotted (even in a square diagram).

\item Redefinition of the components to other species than the
  elements is still not possible.

\item Conditions on state variables like $V, H$ etc are not yet
  implemented.

\item Conditions which are expressions are not implemented.

\item The corrected quasichemical model for liquids is not
  implemented.

\item There is no check on miscibility gaps during a step or map
  command.

\item Saving results from step and map on a file is not possible
  except graphically with GNUPLOT.

\item The mapping is very fragile, lines are frequently missing or
  incomplete.

\item Conditions are not restored after finished step/map.

\item There is no plot of tie-lines.

\item The Scheil-Gulliver solidification model is not implemented.

\end{itemize}

As OC is open source anyone who is interested to implement a
particular feature is welcome to start working on it.

\section{Next release}

Adding the STEP and MAP has shown some problems with the original data
structure.  Thus a complete revision of the data structure will be
made for the version 3.  

All kinds of state variables will be available for conditions and also
expressions of state variables.

The ``dot derivative'' facility will be extended.

The STEP and MAP will be able to handle multicomponent invariant
equilibria and detect miscibility gaps during mapping.

Parallelization will be extended to STEP and MAP calculaions.

\section{Long term goals}

\begin{itemize}
\item A full Fortran/C++ application software interface including the
  use of compatible data structures.

\item An assessment module for model parameters with a lot of help for
  beginners.

\item A teaching package for computational thermodynamics and phase
  diagrams.

\end{itemize}

\begin{thebibliography}{77Zzz}
\bibitem[81Hil]{81Hil} M Hillert, Physica, {\bf 103B} (1981) 31
\bibitem[01Hil]{01Hil} M Hillert, J of Alloys and Comp {\bf 320} (2001) 161
\bibitem[07Luk]{07Luk} H L Lukas, S G Fries and B Sundman, {\em Computational
Thermodynamics}, Cambridge Univ Press (2007)
\bibitem[http://www.opencalphad.org]{ocweb} http://www.opencalphad.org
\bibitem[http://github.com]{github} opencalphad at http://github.com.  
\end{thebibliography}

\newpage

\section{Some diagrams generated with OCv2 macro files}

\begin{figure}[!ht]
\subfigure[\label{fg:s1a}]{
\includegraphics[width=35mm,angle=-90]{figs2/step1np.ps}}
\subfigure[\label{fg:s1b}]{
\includegraphics[width=35mm,angle=-90]{figs2/step1wcr.ps}}
\subfigure[\label{fg:s1c}]{
\includegraphics[width=35mm,angle=-90]{figs2/step1h.ps}}
\subfigure[\label{fg:s1d}]{
\includegraphics[width=35mm,angle=-90]{figs2/step1cp.ps}}
%
\subfigure[\label{fg:s2a}]{
\includegraphics[width=35mm,angle=-90]{figs2/step2g.ps}}
%
\subfigure[\label{fg:s3a}]{
\includegraphics[width=35mm,angle=-90]{figs2/step3y.ps}}
\subfigure[\label{fg:s3b}]{
\includegraphics[width=35mm,angle=-90]{figs2/step3h.ps}}
\subfigure[\label{fg:s3c}]{
\includegraphics[width=35mm,angle=-90]{figs2/step3cp.ps}}
%
\subfigure[\label{fg:s4a}]{
\includegraphics[width=35mm,angle=-90]{figs2/step4g.ps}}
\subfigure[\label{fg:s4b}]{
\includegraphics[width=35mm,angle=-90]{figs2/step4y.ps}}
%
\subfigure[\label{fg:s5a}]{
\includegraphics[width=35mm,angle=-90]{figs2/step5y.ps}}
\subfigure[\label{fg:s5b}]{
\includegraphics[width=35mm,angle=-90]{figs2/step5cp.ps}}
%
%
\caption{In \ref{fg:s1a} to \ref{fg:s1d} diagrams for a high speed
  steel generated with macro file step2-agcu.  In \ref{fg:s2a} Gibbs
  energy curves in Ag-Cu generated with macro file step2-agcu.  In
  \ref{fg:s3a} to \ref{fg:s3c} the speciation of a gas with H and O as
  function of $T$ and its effect on enthalpy and heat capacity
  generated with macro file step3-hogas.  In \ref{fg:s4a} and
  \ref{fg:s4b} the Gibbs energy curves at 400~K in the Fe-Ni system
  for the fcc ordering generated with macro file step4-feni.  The
  variation of ordering in FeNi$_3$ as function of $T$, in
  \ref{fg:s5a} the constituent fractions and in \ref{fg:s5b} the heat
  capacity.  Generated with macro file step5-feni.}
\end{figure}


\begin{figure}
\begin{center}
\subfigure[\label{fg:s6a}]{
\includegraphics[width=35mm,angle=-90]{figs2/step6g.ps}}
%
\subfigure[\label{fg:s7a}]{
\includegraphics[width=35mm,angle=-90]{figs2/step7npt.ps}}
\subfigure[\label{fg:s7b}]{
\includegraphics[width=35mm,angle=-90]{figs2/step7npn.ps}}
\subfigure[\label{fg:s7c}]{
\includegraphics[width=35mm,angle=-90]{figs2/step7pref.ps}}
\subfigure[\label{fg:s7d}]{
\includegraphics[width=35mm,angle=-90]{figs2/step7preb.ps}}
%
\subfigure[\label{fg:m1a}]{
\includegraphics[width=35mm,angle=-90]{figs2/map1pd.ps}}
\subfigure[\label{fg:m1b}]{
\includegraphics[width=35mm,angle=-90]{figs2/map1pdzoom.ps}}
\subfigure[\label{fg:m1c}]{
\includegraphics[width=35mm,angle=-90]{figs2/map1pdrotate.ps}}
\subfigure[\label{fg:m1d}]{
\includegraphics[width=35mm,angle=-90]{figs2/map1ac.ps}}
%
\subfigure[\label{fg:m2a}]{
\includegraphics[width=35mm,angle=-90]{figs2/map2pd.ps}}
\subfigure[\label{fg:m2b}]{
\includegraphics[width=35mm,angle=-90]{figs2/map2pdzoom.ps}}
\end{center}
%
\caption{In \ref{fg:s6a} the Gibbs energy curves for the different
  phases in Fe-Mo at 1400~K, generated with macro file step6-femo.  In
  \ref{fg:s7a} to \ref{fg:s7d} calculated phase amounts as function of
  $T$ or mass fraction of N and ``pitting corrosion equivalents'',
  generated with macro file step7-saf.  In \ref{fg:m1a} to
  \ref{fg:m1d} the phase diagram for the Ag-Cu system generated with
  macro file map1-agcu.  In \ref{fg:m2a} to \ref{fg:m2b} the phase
  diagram for the Cr-Mo system generated with macro file map2-crmo.}
\end{figure}


\begin{figure}
\begin{center}
\subfigure[\label{fg:m3a}]{
\includegraphics[width=35mm,angle=-90]{figs2/map3pd.ps}}
\subfigure[\label{fg:m3b}]{
\includegraphics[width=35mm,angle=-90]{figs2/map3pdzoom.ps}}
%
\subfigure[\label{fg:m4a}]{
\includegraphics[width=35mm,angle=-90]{figs2/map4pd.ps}}
\subfigure[\label{fg:m4b}]{
\includegraphics[width=35mm,angle=-90]{figs2/map4pdzoom.ps}}
%
\subfigure[\label{fg:m5a}]{
\includegraphics[width=35mm,angle=-90]{figs2/map5pd.ps}}
\subfigure[\label{fg:m5b}]{
\includegraphics[width=35mm,angle=-90]{figs2/map5pdzoom.ps}}
%
\subfigure[\label{fg:m6a}]{
\includegraphics[width=35mm,angle=-90]{figs2/map6pd.ps}}
%
\subfigure[\label{fg:m7a}]{
\includegraphics[width=35mm,angle=-90]{figs2/map7pdx.ps}}
%
\subfigure[\label{fg:m8a}]{
\includegraphics[width=35mm,angle=-90]{figs2/map8pd.ps}}
%
\end{center}
%
\caption{ In \ref{fg:m3a} to \ref{fg:m3b} the phase diagram for the
  C-Fe system generated with macro file map3-cfe.  In \ref{fg:m4a} to
  \ref{fg:m4b} the phase diagram for the O-U system generated with
  macro file map4-ou.  In \ref{fg:m5a} to \ref{fg:m5b} the phase
  diagram for the Fe-Mo system generated with macro file map5-femo.
  In \ref{fg:m6a} the isopleth diagram for the Cr-Fe-Ni system at 8
  mass\%Ni generated with macro file map6-ss.  In \ref{fg:m7a} an
  isopleth calclation for a high speed steel generated with macro file
  map7-hss.  As can be seen there are still problems with mapping.  In
  \ref{fg:m8a} a metastable phase diagram for the ordered fcc phase in
  Fe-Ni generated with macro file map8-feni.  }
\end{figure}


\end{document}



