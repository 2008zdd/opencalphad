\documentclass[12pt]{article}
\usepackage[latin1]{inputenc}
\usepackage{graphicx,subfigure}
\topmargin -1mm
\oddsidemargin -1mm
\evensidemargin -1mm
\textwidth 165mm
\textheight 220mm
\parskip 2mm
\parindent 3mm
%\pagestyle{empty}

\begin{document}
\begin{center}
{\Large \bf Installation of OpenCalphad on Linux or macOS}

Bo Sundman, \today

\end{center}

There is no automatic installation routine for OC, you must download
and compile the software yourself.  You may also have to install
Fortran compilers and the GNUPLOT software if you do not already have
them.  The OC development team cannot offer you any help for that,
please ask some local experts if you need help.

The description below applies when installing OC on a ``vanilla''
Linux system, the guides available are:
\begin{itemize}
\item Install-OC-Windows-MinGW
\item Installation de OC sous Windows avec Cygwin (in French)
\item Install-OC-Linux-or-macOS
\end{itemize}

Step by step installation:

\begin{itemize}
\item The code is written in the new Fortran standard and requires a
  compiler like GNU Fortran 4.8 or similar.

\item Normally you have the GNU Fortran compiler in your linux system
  otherwise you can get the GNU Fortran suite free from
  https://SourgeForge.net or some similar site.  If you have Intel or
  another Fortran compiler you must be sure it is compatible with GNU
  Fortran 4.8 or later.

\item open a terminal window and use ``cd'' (change directory) to the
  directory where you unzipped OC.

  \begin{itemize}
  \item run ``make -f Makefile-sequential'' to build a serial version of OpenCalphad...

  \item ... or if you have access to several CPUs you can test OC with
    parallelization using Open MP.  In that case run ``make -f Makefile-parallel'' to build a OpenCalphad
  \end{itemize}

\item {\bf If you have errors running the make command files please
    contact a local expert.}

\item For the graphics you must download and install the free GNUPLOT
  software, for example from SourceForge.

  Make sure your PATH includes the directory with the GNUPLOT program.
  If you do not know how to set your PATH ask a local expert.

\item Creating a home directory for OC

  \begin{itemize}
  \item Create a directory called OCHOME, usually at your home
    directory ``mkdir OCHOME''

  \item Copy the file ochelp.hlp to this directory

  \item Create an environment variable called OCHOME with the path
    to your OCHOME directory.  If you do not know how create an
    environment variable please ask a local expert.

  \item Later you may also add a macro file called ``start.OCM''
    that you want to run every time you start OC on this directory.
    You can also create a subdirectory called ``databases'' that you
    can search when you give the command ``read tdb'' in OC if you
    prefix the file name by ``ocbase/''
  \end{itemize}

\item Look in ``after-installation'' for help using OC.

\end{itemize}

You are welcome to help providing a better installation guide also!

\bigskip

{\large \bf Have fun and help make OC useful!}

\end{document}
